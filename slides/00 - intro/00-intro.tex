%\documentclass[12pt,handout]{beamer}
\documentclass[presentation]{beamer}
\usepackage{../oop-slides-croatti}
\setbeamertemplate{bibliography item}[text]

\newcommand{\lab}{Lab01}

\title[{\lab} -- Introduzione]{Introduzione al laboratorio}

\date[\today]{\today}

\begin{document}

\frame[label=coverpage]{\titlepage}

\begin{frame}<beamer>
	\frametitle{Outline}
	\tableofcontents[]
\end{frame}

\section{Organizzazione del Laboratorio}

\begin{frame}{Organizzazione del Laboratorio}

\begin{itemize}
\item Due turni settimanali
\begin{itemize}
\item Per esigenze di spazi, per garantire un maggiore supporto ai frequentanti, \dots
\end{itemize}
\item Il contenuto della lezione e dell'esecitazione settimanale del laboratorio è il medesimo per entrambi i turni
\item Eventuali variazioni di turno vanno motivate e concordate con il Prof. Viroli (\texttt{mirko.viroli@unibo.it})
\end{itemize}

\begin{block}{Primo Turno (cognomi nell'intervallo [A-Gim])}
\begin{itemize}
\item Lunedì, 9:00 - 13:00
\item Lab. 2.2, Campus Cesena
\end{itemize}
\end{block}

\begin{block}{Secondo Turno (cognomi nell'intervallo [Gin-Z])}
\begin{itemize}
\item Martedì, 13:00 - 17:00
\item Lab. 2.2, Campus Cesena
\end{itemize}
\end{block}

\end{frame}

\begin{frame}{Docenti del Laboratorio}

\begin{block}{Prof. Danilo Pianini -- Responsabile Modulo 1}
\begin{itemize}
\item mail: \texttt{danilo.pianini@unibo.it}
\item ricevimento: Martedì, 17:00 - 18:00
\end{itemize}
\end{block}

\begin{block}{Prof. Angelo Croatti -- Responsabile Modulo 2}
\begin{itemize}
\item mail: \texttt{a.croatti@unibo.it}
\item ricevimento: su appuntamento, da concordare via mail
\end{itemize}
\end{block}

\begin{block}{Ing. Roberto Casadei -- Tutor Didattico}
\begin{itemize}
\item mail: \texttt{roby.casadei@unibo.it}
\end{itemize}
\end{block}

\end{frame}

\section{Forum e supporto}

\begin{frame}{Il Forum del Corso}

\begin{block}{}
\begin{itemize}
\item Il dubbio di uno studente, probabilmente, è anche il dubbio di qualcun'altro (\textbf{condivisione}) 
\item Anche gli studenti possono rispondere ai post di altri studenti (\textbf{discussione})
\item Utilizzare frequentemente il forum a supporto dei propri colleghi è valutato positivamente
\end{itemize}
\end{block}
\vfill
\begin{itemize}
\item Per chiarimenti, ulteriori delucidazioni e spiegazioni si incoraggia l'uso del \textbf{Forum del Corso}
\begin{itemize}
\item link accessibile dal sito del corso (IOL) \texttt{http://bit.ly/oop19-cesena}
\item da preferire all'email inviata direttamente al/ai docente/i
\end{itemize}
\item L'email resta il canale da utilizzare per comunicazioni \textbf{confidenziali}
\begin{itemize}
\item con l'accortenzza di mettere sempre in copia tutti i docenti del corso
\end{itemize}
\end{itemize}



\end{frame}

\begin{frame}{Il Laboratorio}

\begin{itemize}
\item Consente di mettere in pratica quanto visto nelle lezioni in aula
\begin{itemize}
\item con il supporto diretto del docente e del tutor
\end{itemize}
\item Integra ed espande i contenuti affrontati in aula
\item Introduce strumenti e metodologie (non affrontati in aula)
\begin{itemize}
\item Development Tools, IDE
\item Metodologie di gestione dei progetti
\item Librerie a supporto dello sviluppo
\item \dots
\end{itemize}
\end{itemize}

\begin{block}{Organizzazione di ciascun turno di laboratorio}
\begin{enumerate}
\item Lezione Frontale (30-60 min)
\begin{itemize}
\item Introduce nuovi concetti non visti in aula
\end{itemize}
\item Esercitazione
\begin{itemize}
\item Un set di esercizi da svolgere in autonomia\dots
\item \dots con il supporto del docente e del tutor!
\end{itemize}
\end{enumerate}
\end{block}

\end{frame}

\begin{frame}{Svolgimento di ciascun esercizio}
\begin{enumerate}\itemsep20pt
\item Lettura della consegna
\begin{itemize}
\item Chiedere al docente/tutor in caso di dubbi
\end{itemize}
\item Svolgimento dell'esercizio
\begin{itemize}
\item Chiedere al docente/tutor in caso di dubbi e/o difficoltà
\end{itemize}
\item Segnalazione al docente/tutor del avvenuto completamento
\begin{itemize}
\item La ``correzione'' è fondamentale per ricevere commenti dal docente/tutor per migliorare la propria soluzione o l'approccio risolutivo
\end{itemize}
\end{enumerate}
\end{frame}

\section{Contenuti e Obiettivi}

\begin{frame}{Overview sui contenuti}

\begin{enumerate}\itemsep10pt
\item Tooling Java
\item Eclipse IDE, debug tools
\item Controllo di versione
\item Testing, documentazione, deployment
\item Qualità del codice
\item Programmazione multipiattaforma
\item Profiling tools
\item JavaFX
\item C\# IDE e tools
\end{enumerate}

\end{frame}

\begin{frame}{Obiettivi del Laboratorio}
\begin{itemize}\itemsep10pt
\item Acquisire le completenze necessarie per:
\begin{enumerate}
 \item diventare ottimi programmatori
 \item diventare buoni progettisti
\end{enumerate}
\item Fondamentale: \textbf{mettersi in gioco e impegnarsi!}
\begin{itemize}
\item percorso a difficoltà crescente (codice e consegne in inglese, tool avanzati, \dots)
\item richiede: attenzione in aula, attenzione e impegno in laboratorio, studio autonomo
\end{itemize}
\end{itemize}
\end{frame}
  
%\section{Java Overview}
%
%%====================
%%Architettura a Runtime
%%====================
%\fr{Architettura a Runtime}{
%	\fg{height=0.80\textheight}{img/arch_runtime.png}
%}
%
%
%
%
%%====================
%%JDK
%%====================
%
%\fr{Java Development Kit (JDK)}{
%	
%	\bl{Esistono diverse distribuzioni di Java}{
%		\begin{description}
%			\item[JRE] -- Java Runtime Environment: esecuzione di programmi
%			\item[JDK] -- JRE + tool per lo sviluppo di programmi (e.g. compilatore)
%			\item[J2EE] -- Java Enterprise Edition: JDK + set esteso di librerie
%		\end{description}
%	}
%	
%	\bl{Noi faremo riferimento al JDK}{Include il necessario per eseguire applicazioni
%	Java (JRE), i tool (e molte librerie) per sviluppare applicazioni,
%	e la relativa  documentazione}
%	
%	\bl{I Principali Tool}{
%		\begin{description}
%			\item[javac] -- Compilatore Java
%			\item[java] -- Java virtual machine, per eseguire applicazioni Java
%			\item[javadoc] -- Per creare automaticamente documentazione in HTML
%			\item[jar] -- Creazione di archivi compressi (anche eseguibili) contenenti bytecode e risorse
%		\end{description}
%	}
%}
%
%\section{Richiamo: il file system}
%\begin{frame}{Elementi base del file system}
%  \begin{itemize}
%    \item I sistemi operativi odierni consentono di memorizzare permanentemente le informazioni su 
%supporti di memorizzazione di massa (dischi magnetici, dispositivi a stato solido...)
%    \item Le informazioni su questi supporti sono organizzate in file e cartelle:
%      \begin{itemize}
%        \item i file contengono le informazioni
%        \item le cartelle sono contenitori, all'interno contengono i file ed altre cartelle
%      \end{itemize}
%    \item La cartella più esterna, che contiene tutte le altre, è detta root. Essa rappresenta il livello gerarchico più alto del file system
%      \begin{itemize}
%        \item In *nix (Linux, MacOS, BSD, Solaris...), vi è una unica radice, ossia \texttt{/}
%        \item In Windows, c'è una root per file system identificata da una lettera di unità (e.g. 
%\texttt{C:}, \texttt{D:})
%      \end{itemize}
%    \item La stringa che descrive un intero percorso dalla root fino ad un elemento del file system
%prende il nome di \emph{percorso}, e.g.:
%    \begin{itemize}
%        \item \texttt{/home/user/frameworkFS.jar} (percorso Unix)
%        \item \texttt{C:\textbackslash{}Windows\textbackslash{}win32.dll} (percorso Windows)
%    \end{itemize}
%  \end{itemize}
%\end{frame}
%
%\begin{frame}[fragile]{Manipolare il file system}
%  L'utente può osservare e manipolare il file system:
%  \begin{itemize}
%    \item sapere quali file e cartelle contiene una cartella
%    \item creare nuovi file e cartelle
%    \item spostare file e cartelle dentro altre cartelle
%    \item rinominare file e cartelle
%    \item eliminare file e cartelle
%  \end{itemize}
%  Il software che consente di osservare e manipolare il file system prende il nome di \alert{file manager}.
%  \begin{itemize}
%    \item Su Windows, esso è ``Esplora risorse'' (\texttt{explorer.exe})
%    \item Su MacOS, il principale è ``Finder''
%    \item Su Linux (e Android) ne esistono diversi (Nautilus, Dolphin, Thunar, Astro...)
%  \end{itemize}
%\end{frame}
%
%
%
%
%\section{Richiamo: Interprete Comandi}
%\fr{Interprete Comandi}{
%  \bl{}{
%    Programma che permette di interagire con il S.O. mediante comandi impartiti in modalità testuale (non grafica), via linea di comando
%  }
%% 	\bl{Scripting}{
%% 		L'interprete comandi può avere un linguaggio associato con cui è possibile scrivere script
%% %		\iz{
%% %			\item Utili (principalmente) per automatizzare task di diversa natura
%% %		}
%% 	}
%% 	\bl{Vari tipi di comandi}{
%% 		\iz{
%% 			\item Navigazione file system
%% 			\item Interazione con il file system
%% 			\item Esecuzione di programmi da riga di comando
%% 			\item ...
%% 		}
%% 	}
%  \begin{itemize}
%    \item Nell'antichità (in termini informatici) le interfacce grafiche erano sostanzialmente inesistenti, e l'interazione con i calcolatori avveniva di norma tramite interfaccia testuale
%    \item Tutt'oggi, le interfacce testuali sono utilizzate:
%    \begin{itemize}
%      \item per automatizzare le operazioni
%      \item per velocizzare le operazioni (scrivere un comando è spesso molto più veloce di andare a fare click col mouse in giro per lo schermo)
%      \item per fare operazioni complesse con pochi semplici comandi
%      \item non tutti i software sono dotati di interfaccia grafica
%      \item alcune opzioni di configurazione del sistema operativo restano accessibili solo via linea di comando
%      \begin{itemize}
%        \item (anche su Windows: ad esempio i comandi per associare le estensioni ad un eseguibile)
%      \end{itemize}
%    \end{itemize}
%  \end{itemize}
%  \bl{}{
%    Lo vedrete in maniera esaustiva nel corso di Sistemi Operativi...
%  }
%}
%
%\fr{Sistemi *nix (Linux, MacOS X, FreeBSD, Minix...)}{
%        \bl{Nei sistemi UNIX esistono vari tipi di interpreti, chiamati shell}{
%                Alcuni esempi
%                \iz{
%                        \item Bourne shell (sh)
%                                \iz{ \item Prima shell sviluppata per UNIX (1977)}
%                        \item C-Shell (csh)
%                                \iz{ \item Sviluppata da Bill Joy per BSD}
%                        \item Bourne Again Shell (bash)
%                                \iz{ \item Parte del progetto GNU, è un super set di Bourne shell}
%                        \item ...
%                }
%        }
%        \bl{Per una panoramica completa delle differenze}{
%                \textcolor{blue}{\url{http://www.faqs.org/faqs/unix-faq/shell/shell-differences/}}
%        }
%}
%
%\fr{Sistemi Windows}{
%  \bl{}{
%    L'interprete comandi è rappresentato dal programma \texttt{cmd.exe} in \texttt{C:$\backslash$Windows$\backslash$System32$\backslash$cmd.exe}
%    \iz{
%      \item Eredita in realtà sintassi e funzionalità della maggior parte dei comandi del vecchio MSDOS
%      \item Versioni recenti hanno introdotto PowerShell, basato su .NET e C\#
%      \item Windows 10 ha introdotto il supporto a bash tramite Linux Subsystem for Windows
%    }
%  }
%  \fg{height=0.4\textheight}{img/prompt.png}
%}
%
%\begin{frame}[fragile]{Aprire un terminale in laboratorio}
%  \begin{itemize}
%    \item In laboratorio, troverete il terminale (prompt dei comandi) clickando su Start $\Rightarrow$ Programmi $\Rightarrow$ Accessori $\Rightarrow$ Prompt dei comandi
%    \item Metodo più rapido: Start  $\Rightarrow$ Nella barra di ricerca, digitare \texttt{cmd} $\Rightarrow$ clickare su \texttt{cmd.exe}
%  \end{itemize}
%\end{frame}
%
%\begin{frame}[fragile]{File system e terminale: cheat sheet}
%\label{slide:commands}
%  \begin{center}
%    \begin{tabular}{| l | c | c |}
%      \hline
%      \textbf{Operazione} & \textbf{Comando Unix} & \textbf{Comando Win} \\ \hline
%      \scriptsize{}Visualizzare la directory corrente & \texttt{pwd} & \texttt{echo \%cd\%}  \\ \hline
%      \scriptsize{}Eliminare il file \texttt{f} (non va con le cartelle!) & \texttt{rm f} & \texttt{del f} \\ \hline
%      \scriptsize{}Eliminare la directory \texttt{nd} & \texttt{rm -r nd} & \texttt{rd nd} \\ \hline
%      \scriptsize{}Contenuto della directory corrente & \texttt{ls -alh} & \texttt{dir} \\ \hline
%%       \scriptsize{}Avviare un eseguibile di nome \texttt{pippo} & \texttt{./p} & \texttt{.\textbackslash{}p} \\ \hline
%      \scriptsize{}Cambiare unità disco (passare a D:) & -- & \texttt{D:} \\ \hline
%      \scriptsize{}Passare alla directory \texttt{nd} & \texttt{cd nd} & \texttt{cd nd} \\ \hline
%      \scriptsize{}Passare alla directory di livello superiore & \texttt{cd ..} & \texttt{cd..} \\ \hline
%      \scriptsize{}Spostare (rinominare) un file \texttt{f1} in \texttt{f2} & \texttt{mv f1 f2} & \texttt{move f1 f2} \\ \hline
%      \scriptsize{}Copiare il file \texttt{f} in \texttt{fc} & \texttt{cp f fc} & \texttt{copy f fc} \\ \hline
%      \scriptsize{}Creare la directory \texttt{d} & \texttt{mkdir d} & \texttt{md d} \\ \hline
%    \end{tabular}
%  \end{center}
%  Eseguire delle prove ed esser certi di aver compreso come utilizzare ogni comando. Per \emph{cominciare} l'esame, in particolare, dovrete usare il comando \texttt{cd}: siate certi di aver capito cosa fa!
%\end{frame}
%
%\begin{frame}[fragile]{Uso intelligente del terminale}
%  \begin{block}{Autocompletamento}
%    \scriptsize{}
%    Sia *nix che Windows offrono la possibilità di effettuare autocompletamento, ossia chiedere al sistema di provare a completare un comando. Per farlo si utilizza il tasto ``tab'' (quello con due frecce orientate in maniera opposta, sopra il lucchetto).
%  \end{block}
%  \begin{block}{Memoria dei comandi precendenti}
%    \scriptsize{}
%    Sia *nix che Windows offrono la possibilità di richiamare rapidamente i comandi inviati precedentemente premendo il tasto ``freccia su''. I sistemi *nix supportano anche il lancio di comandi eseguiti in sessioni precedenti (non perde memoria col riavvio del terminale). 
%  \end{block}
%  \begin{block}{Interruzione di un programma}
%    \scriptsize{}
%    È possibile interrompere forzatamente un programma (ad esempio perché inloopato). Per farlo, sia su Windows che in *nix, si prema ctrl+c.
%  \end{block}
%  \begin{block}{Ricerca nella storia dei comandi precedenti}
%    \scriptsize{}
%    Premendo ctrl+r seguito da un testo da cercare, i sistemi *nix supportano la ricerca all'interno dei comandi lanciati recentemente, anche in sessioni utente precedenti. Non disponibile su Windows.
%  \end{block}
%\end{frame}
%
%
%
%\section{Preparazione dell'ambiente di lavoro}
%
%\fr{Preparazione Ambiente di Lavoro 1/3}{
%	\iz{
%		\item Accendere il PC
%		\item Loggarsi sul sito del corso
%		\iz{
%			\item \textcolor{blue}{\url{http://bit.ly/unibooop2018}}
%		}
%		\item Scaricare dalla sezione dedicata a questa settimana il materiale dell'esercitazione odierna
%		\item Spostare il file scaricato sul Desktop
%		\item Decomprimere il file usando 7zip (o un programma analogo) sul Desktop
%	}
%}
%
%
%\fr{Preparazione Ambiente di Lavoro 2/3}{
%	\iz{
%		\item Aprire il prompt dei comandi 
%	}	
%	\fg{height=0.8\textheight}{img/open_prompt.png}
%}
%
%\fr{Preparazione Ambiente di Lavoro 3/3}{
%	\iz{
%		\item Posizionarsi all'interno della cartella scompattata con l'ausilio del comando \texttt{cd} (change directory)
%		\en{
%			\item \texttt{cd Desktop} e premere invio, dopodiché
%			\item \texttt{cd lab01} e premere invio
%		}
%	}
%	\fg{height=0.4\textheight}{img/cd_prompt.png}
%}
%
%\section{Compilazione ed Esecuzione da Riga di Comando}
%
%\fr{Compilazione ed Esecuzione, comandi di base}{
%	\bl{Compilazione}{
%		Compilazione di una classe (comando \textcolor{blue}{javac})
%		\iz{
%			\item \alert{\cil{javac NomeFileSorgente.java}} 
%			\item \alert{\cil{javac *.java}} compila tutti i sorgenti nella directory corrente
%		}
%	}
%	\bl{Esecuzione}{
%		Esecuzione di un programma Java (comando \textcolor{blue}{java})
%		\iz{
%			\item	\alert{\cil{java NomeDellaClasse}}
%		}
%	}
%	Si noti che il compilatore \emph{traduce} \textbf{file} sorgente in \textbf{file} binari, 
%mentre \\
%    L'interprete Java (la Java Virtual Machine) esegue una ed una sola \textbf{classe}.
%    
%    Il compilatore Java lavora su \textit{file}, l'interprete Java su \textit{classi}
%}
%
%\fr{Appendice: operazioni coi complessi}{
%	\bl{Operazioni sui numeri complessi --- Breve ripasso}{
%		Siano $z,w \in \mathbb{C} : z = a + ib,\ w = c + id$, allora:
%		\iz {
%			\item Confronto: $z = w \iff a = c \wedge b = d$
%			\item Somma algebrica: $z \pm w = a \pm c + i(b \pm d)$
%			\item Prodotto: $z \cdot w = (a+ib)(c+id) = ac-bd+i(bc+ad)$
%			\item Rapporto: $\frac{z}{w} = \frac{(a+ib)(c-id)}{(c+id)(c-id)} = \frac{ac + bd}{c^2 + 
%d^2} + i\frac{bc - ad}{c^2 + 
%d^2}$
%		}
%	}	
%}

\end{document}

\begin{frame}[allowframebreaks]
 \frametitle{Bibliography}
	\bibliographystyle{plain}
	\small
 \bibliography{biblio}
\end{frame}

