\documentclass[presentation]{beamer}
\usepackage{../oop-slides-pianini}
\usepackage{url}
\setbeamertemplate{bibliography item}[text]
\title[OOP04 -- Ereditarietà]{04 \\Uso avanzato di Eclipse, ereditarietà e classi astratte}
\author[Pianini]{Danilo Pianini \\ Giovanni Ciatto, Angelo Croatti, Mirko Viroli}

\begin{document}

\frame[label=coverpage]{\titlepage}

%====================
%Outline
%====================
\fr{Outline}{
\bl{Goal della lezione}{
\iz{ 
\item Utilizzo avanzato di Eclipse
\iz{
\item Debugging
\item Refactoring
\item Keyboard Shortcuts
}
\item Imparare a costruire classi sfruttando l'ereditarietà
\iz{
\item Scrittura ed estensione di (1) classi e (2) classi astratte
}
\item Ragionamenti sul (buon) design di Software OOP-oriented
}
}
}

\section{Eclipse: Aspetti Avanzati}

\section{Lab Startup}

\fr{Preparazione Ambiente di Lavoro 1/2}{
  \iz{
    \item Accendere il PC
    \item Loggarsi sul sito del corso
    \iz{
      \item\textcolor{blue}{\url{https://bit.ly/oop2016cesena}}
    }
    \item Scaricare dalla sezione \texttt{lab} del sito il file \texttt{lab04.zip} contenente il materiale dell'esercitazione odierna
    \item Spostare il file scaricato sul Desktop
    \item Decomprimere il file usando 7zip (o un programma analogo) sul Desktop
    \item Copiare la cartella scompattata (\texttt{lab04}) nel vostro workspace di Eclipse
    \iz{
      \item p.e. \texttt{C:$\backslash$Users$\backslash$<username>$\backslash$workspace}
    }
    \item Importare il progetto \texttt{lab04} con la procedura di importazione dei progetti vista durante lo scorso laboratorio
  }
  
}


\fr{Modalità di Lavoro}{
  \bl{}{
    \en{
      \item Leggere la consegna
      \item Risolvere l'esercizio in autonomia
      \item Cercare di risolvere autonomamente eventuali piccoli problemi che possono verificarsi durante lo svolgimento degli esercizi
      \item \textcolor{red}{Utilizzare le funzioni di test presenti nei sorgenti per il testing dell'esercizio}
      \item Contattare i docenti nel caso vi troviate a lungo bloccati nella risoluzione di uno specifico esercizio
      \item A esercizio ultimato contattare i docenti per un rapido controllo della soluzione realizzata
%       \item \textcolor{red}{Scrivere la Javadoc per l'esercizio svolto} (o in lab o a casa in mancanza di tempo)
      \item Proseguire con l'esercizio seguente
    }
  }
}

\section{Ereditarietà}

\subsection{Refactoring di un esercizio precedente}

\fr{SimpleBankAccount e StrictBankAccount}{
  \bl{Miglioriamo la precedente esercitazione}{
    \iz{
      \item Nelle soluzioni dell'esercizio della volta scorsa, c'è molto codice duplicato fra \texttt{SimpleBankAccount} e \texttt{StrictBankAccount}
      \item Si crei la classe \texttt{ExtendedStrictBankAccount extends SimpleBankAccount}, con lo stesso comportamento di StrictBankAccount, cercando di ottimizzare il riuso di codice.
		\item È \textbf{proibito} apportare modifiche a \texttt{SimpleBankAccount}, di qualunque tipo.
		\item Si mostri il risultato al docente per una correzione
    }
  }
  \bl{Una nuova architettura}{
    \iz{
      \item La soluzione precedente è stata ottenuta per modifica di un design sub-ottimo (non conoscevamo ancora l'ereditarietà!)
      \item Come sarebbe stato il design dell'applicazione se avessimo conosciuto fin dall'inizio il meccanismo di ereditarietà?
    }
  }
}
  
\fr{SimpleBankAccount e StrictBankAccount}{
  \bl{Una nuova architettura}{
    \iz{
      \item All'interno del package \texttt{it.unibo.oop.lab04.bank2}, si creino le seguenti classi:
      \iz{
        \item \texttt{abstract AbstractBankAccount implements BankAccount}
        \iz{
          \item dovrà contenere elementi comuni a \textbf{tutti} i conti correnti
          \item \textbf{tutti} i campi della classe dovranno essere privati
          \item la classe dovrà esporre come metodi \texttt{public} tutti e soli i metodi dell'interfaccia \texttt{BankAccount}
          \item deve definire  ed utilizzare due metodi \texttt{protected abstract}: \texttt{boolean isWithdrawAllowed(double)} (\texttt{true} se se è possibile prelevare dal conto il valore passato) e \texttt{double computeFee()} (ritorna l'ammontare complessivo dei costi di gestione)
        }
        \item \texttt{ClassicBankAccount extends AbstractBankAccount}
        \iz{
          \item Implementa lo stesso comportamento di \texttt{SimpleBankAccount}
        }
        \item \texttt{RestrictedBankAccount extends AbstractBankAccount}
        \iz{
          \item Implementa lo stesso comportamento di \texttt{StrictBankAccount}
        }
      }
      \item In cosa è migliore della soluzione precedente?
    }
  }
}

\subsection{Nuovo scenario: robot}

\fr{Costruzione e Testing di Robot}{
  \bl{}{
    Nella lezione di oggi studieremo in pratica l'ereditarietà e le classi astratte usando come scenario di riferimento la modellazione e il testing di robot
  }
  
  \iz{ \item \cil{package it.unibo.oop.lab04.robot.base}: set di interfacce e classi fornite }
    \iz{
      \item \cil{Robot}
      \item \cil{Position2D}
      \item \cil{BaseRobot}
      \item \cil{RobotPosition}
      \item \cil{RobotEnvironment}
    }
  Si utilizzi \texttt{TestRobots} per capire il funzionamento delle classi suddette
}

\fr{Costruzione e Testing di Robot}{
  \fg{width=0.99\textwidth}{img/robots_core.png}
}

\subsection{Robot con braccia}

\fr{Suggested design}{
  \fg{width=0.99\textwidth}{img/arms}
}

\fr{Design guidato}{
  Sfruttando le classi base già definite, e facendo riferimento allo schema UML allegato nella slide precedente, si realizzino nel \cil{package it.unibo.oop.lab04.robot.arms}:
  \iz{
    \scriptsize
    \item \cil{interface RobotWithArms}: descrive un robot dotato di braccia. Presenta tre metodi:
    \iz{
      \scriptsize
      \item \cil{boolean pickUp()} raccoglie un oggetto, restituisce \cil{true} se ha successo
      \item \cil{boolean dropDown()} rilascia un oggetto, restituisce \cil{true} se ha successo
      \item \cil{int getItemsCarried()} restituisce il numero di oggetti che si stanno trasportando.
    }
    \item \cil{class BasicArm}: modella un singolo braccio robot. Si faccia riferimento allo schema UML precedente, sapendo che ciascun braccio è in grado di sollevare un solo oggetto alla volta, che ha un consumo per prendere un oggetto e per lasciarlo.
    \item \cil{class RobotWithTwoArms extends BaseRobot implements RobotWithArms}: modella un robot con due braccia. Al comando \cil{pickUp()}, il robot preleva un oggetto, aumentando il numero di oggetti che sta trasportando, ed occupando una delle braccia. Quando tutte le braccia sono occupate, \cil{pickUp()} fallisce (restituisce \cil{false}). Allo stesso modo, quando le braccia sono vuote, \cil{dropDown()} fallisce. Quando il robot sta trasportando oggetti, il suo consumo di batteria è superiore.
  }
  Si utilizzi \texttt{TestRobots} per capire il funzionamento delle classi suddette
}

\section{Design}

\fr{Design di una applicazione}{
  \iz {
    \item Come ingegneri, dovrete essere in grado di tradurre una descrizione in linguaggio naturale (``a parole'') di una applicazione in termini di astrazioni adatte ad essere scritte in forma di software (nel nostro caso, in elementi di linguaggio di programmazione orientato agli oggetti)
    \item Si tratta dell'esercizio \textbf{più difficile} e \textbf{più importante} che dovrete fare quando realizzerete il progetto
    \item Dalle vostre capacità di analisi del problema e di design della soluzione dipende il successo o meno del vostro software
    \item Nel prossimo esercizio, vi sarà data una descrizione in linguaggio naturale, e starà a voi realizzare il software autonomamente
    \item \textbf{SUGGERIMENTO} --- Prima di iniziare a ``sporcarsi le mani'' col codice, prendete carta e penna e cercate di ottenere un design chiaro dei componenti che realizzerete: nessuno costruisce un'automobile partendo con l'assemblare i bulloni, ma decide prima quali saranno i componenti e come andranno connessi fra loro.
  }
}

\subsection{Robot componibili}

\fr{Requisiti: robot componibile}{
  \iz {
    \item Si desidera realizzare un robot di tipo componibile.
    \item Un robot componibile è un robot al quale possono essere aggiunti o rimossi nuovi componenti arbitrari.
    \item Il robot componibile espone una funzionalità che consente di mettere in moto e far funzionare tutti i componenti connessi, a patto ovviamente che siano accesi.
    \item Quando tale funzionalità viene chiamata, il robot mette in funzione in ordine, per una sola volta, tutti i componenti ad esso connessi.
  }
}

\fr{Requisiti: componenti di robot}{
  \iz {
    \item Un componente per robot è un'entità che può essere accesa o spenta.
    \item Il componente può essere non connesso oppure connesso ad un solo robot.
    \item Il componente è in grado di compiere operazioni sul robot.
    \item Ciascun componente ha un proprio consumo di energia.
    \item Quando il robot a componenti fa uso del componente, deve scalare il consumo di energia del componente dalla propria batteria.
    \item Alcuni componenti sono in grado di supportare dei comandi.
    \item Ciascun componente comandabile ha il proprio set di comandi
    \item Il componente comandabile può ricevere un comando
    \item Alla ricezione del comando, se il comando corrisponde ad uno dei comandi supportati, il componente comandabile cambia il proprio comportamento in maniera tale da eseguire il comando richiesto.
  }
}

\fr{Requisiti: test}{
  Si desidera testare l'infrastruttura creando un robot componibile ed assegnandogli i seguenti componenti:
  \iz {
    \item Batteria atomica:
    \iz {
      \scriptsize
      \item Componente non comandabile.
      \item Batteria alimentata ad Uranio-239, che ricarica istantaneamente il robot.
      \item In fase di test, per evitare il surriscaldamento, si attivi la batteria atomica solo nel caso in cui la batteria del robot sia al di sotto del 50\%.
    }
    \item Navigatore di confine:
    \iz {
      \scriptsize
      \item Componente non comandabile.
      \item Una volta avviato, fa sì che il robot raggiunga il bordo del \texttt{RobotEnvironment} e continui ad esplorarlo.
    }
    \item Due braccia prensili:
    \iz {
      \scriptsize
      \item Componente comandabile, con comandi ``pick'' e ``drop''.
      \item Se è attivo il comando ``pick'' e il braccio non ha alcun oggetto in mano, allora ne viene preso uno.
      \item Se è attivo il comando ``pick'' e il braccio ha già un oggetto in mano, allora non viene fatta alcuna azione.
      \item Se è attivo il comando ``drop'' e il braccio ha già un oggetto in mano, allora l'oggetto viene lasciato.
      \item Se è attivo il comando ``drop'' e il braccio non ha alcun alcun oggetto in mano, allora non viene fatta alcuna azione.
    }
  }
}

%\fr{Badge: Pimp by bot} {
%  \iz {
%    \item Chi riesce a completare un robot componibile dotato di tutte le componenti sopra indicate può creare un nuovo componente personalizzato.
%    \item I componenti innovativi e di buona qualità che verranno postati sul forum studenti vinceranno un badge del corso.
%  }
%}


\begin{frame}[allowframebreaks]
 \frametitle{Bibliography}
  \bibliographystyle{plain}
  \small
 \bibliography{biblio}
\end{frame}


\end{document}

