%\documentclass[12pt,handout]{beamer}
\documentclass[presentation]{beamer}
\usepackage{../oop-slides-croatti}
\setbeamertemplate{bibliography item}[text]

\newcommand{\lab}{Lab01}

\title[{\lab} -- Strumenti di Base]{Strumenti di Base}

\date[\today]{\today}

\begin{document}

\frame[label=coverpage]{\titlepage}

\begin{frame}<beamer>
	\frametitle{Outline}
	\tableofcontents[]
\end{frame}
  
\section{Java Development Kit (JDK) Overview}


\fr{Java Development Kit (JDK)}{
	
	\bl{Esistono diverse distribuzioni di Java}{
		\begin{description}
			\item[JRE] -- Java Runtime Environment: esecuzione di programmi
			\item[JDK] -- JRE + tool per lo sviluppo di programmi (e.g. compilatore)
			\item[J2EE] -- Java Enterprise Edition: JDK + set esteso di librerie
		\end{description}
	}
	
	\bl{Noi faremo riferimento al JDK}{Include il necessario per eseguire applicazioni
	Java (JRE), i tool (e molte librerie) per sviluppare applicazioni,
	e la relativa  documentazione}
	
	\bl{I Principali Tool del JDK}{
		\begin{description}
			\item[javac] -- Compilatore Java
			\item[java] -- Java virtual machine, per eseguire applicazioni Java
			\item[javadoc] -- Per creare automaticamente documentazione in HTML
			\item[jar] -- Creazione di archivi compressi (anche eseguibili) contenenti bytecode e risorse
		\end{description}
	}
}

\fr{Java Development Kit (JDK)}{

\begin{itemize}
\item Java 11 -- ultima versione LTS del JDK
\begin{itemize}
\item Esistono varie versioni e vari distributori (OpenJDK, Oracle JDK, Eclipse OpenJ9, Amazon, \dots)
\item OpenJDK -- implementazione non commerciale, riferimento de facto
\end{itemize}
\end{itemize}

\begin{block}{Riferimenti per il corso}
\begin{itemize}
\item Java 11, distribuita da OpenJDK
\item AdoptOpenJDK: riferimento per il download del JDK
\begin{itemize}
\item \texttt{https://adoptopenjdk.net}
\end{itemize}
\end{itemize}
\end{block}
}

\begin{frame}{AdoptOpenJDK}
\fg{height=0.80\textheight}{img/adoptopenjdk.jpg}
\end{frame}


%====================
%Architettura a Runtime
%====================
\fr{Java Virtual Machine: Architettura a Runtime}{
	\fg{height=0.80\textheight}{img/arch_runtime.png}
}

\section{Il file system e l'interprete dei comandi (richiamo)}

\begin{frame}{Elementi base del file system}
  \begin{itemize}
    \item I sistemi operativi odierni consentono di memorizzare permanentemente le informazioni su 
supporti di memorizzazione di massa (dischi magnetici, dispositivi a stato solido...)
    \item Le informazioni su questi supporti sono organizzate in file e cartelle:
      \begin{itemize}
        \item i file contengono le informazioni
        \item le cartelle sono contenitori, all'interno contengono i file ed altre cartelle
      \end{itemize}
    \item La cartella più esterna, che contiene tutte le altre, è detta root. Essa rappresenta il livello gerarchico più alto del file system
      \begin{itemize}
        \item In *nix (Linux, MacOS, BSD, Solaris...), vi è una unica radice, ossia \texttt{/}
        \item In Windows, c'è una root per file system identificata da una lettera di unità (e.g. 
\texttt{C:}, \texttt{D:})
      \end{itemize}
    \item La stringa che descrive un intero percorso dalla root fino ad un elemento del file system
prende il nome di \emph{percorso}, e.g.:
    \begin{itemize}
        \item \texttt{/home/user/frameworkFS.jar} (percorso Unix)
        \item \texttt{C:\textbackslash{}Windows\textbackslash{}win32.dll} (percorso Windows)
    \end{itemize}
  \end{itemize}
\end{frame}

\begin{frame}[fragile]{Manipolare il file system}
  L'utente può osservare e manipolare il file system:
  \begin{itemize}
    \item sapere quali file e cartelle contiene una cartella
    \item creare nuovi file e cartelle
    \item spostare file e cartelle dentro altre cartelle
    \item rinominare file e cartelle
    \item eliminare file e cartelle
  \end{itemize}
  Il software che consente di osservare e manipolare il file system prende il nome di \alert{file manager}.
  \begin{itemize}
    \item Su Windows, esso è ``Esplora risorse'' (\texttt{explorer.exe})
    \item Su MacOS, il principale è ``Finder''
    \item Su Linux (e Android) ne esistono diversi (Nautilus, Dolphin, Thunar, Astro...)
  \end{itemize}
\end{frame}

\fr{Interprete Comandi}{
  \bl{}{
    Programma che permette di interagire con il S.O. mediante comandi impartiti in modalità testuale (non grafica), via linea di comando
  }
% 	\bl{Scripting}{
% 		L'interprete comandi può avere un linguaggio associato con cui è possibile scrivere script
% %		\iz{
% %			\item Utili (principalmente) per automatizzare task di diversa natura
% %		}
% 	}
% 	\bl{Vari tipi di comandi}{
% 		\iz{
% 			\item Navigazione file system
% 			\item Interazione con il file system
% 			\item Esecuzione di programmi da riga di comando
% 			\item ...
% 		}
% 	}
  \begin{itemize}
    \item Nell'antichità (in termini informatici) le interfacce grafiche erano sostanzialmente inesistenti, e l'interazione con i calcolatori avveniva di norma tramite interfaccia testuale
    \item Tutt'oggi, le interfacce testuali sono utilizzate:
    \begin{itemize}
      \item per automatizzare le operazioni
      \item per velocizzare le operazioni (scrivere un comando è spesso molto più veloce di andare a fare click col mouse in giro per lo schermo)
      \item per fare operazioni complesse con pochi semplici comandi
      \item non tutti i software sono dotati di interfaccia grafica
      \item alcune opzioni di configurazione del sistema operativo restano accessibili solo via linea di comando
      \begin{itemize}
        \item (anche su Windows: ad esempio i comandi per associare le estensioni ad un eseguibile)
      \end{itemize}
    \end{itemize}
  \end{itemize}
  \bl{}{
    Lo vedrete in maniera esaustiva nel corso di Sistemi Operativi...
  }
}

\fr{Sistemi *nix (Linux, MacOS X, FreeBSD, Minix...)}{
        \bl{Nei sistemi UNIX esistono vari tipi di interpreti, chiamati shell}{
                Alcuni esempi
                \iz{
                        \item Bourne shell (sh)
                                \iz{ \item Prima shell sviluppata per UNIX (1977)}
                        \item C-Shell (csh)
                                \iz{ \item Sviluppata da Bill Joy per BSD}
                        \item Bourne Again Shell (bash)
                                \iz{ \item Parte del progetto GNU, è un super set di Bourne shell}
                        \item ...
                }
        }
        \bl{Per una panoramica completa delle differenze}{
                \textcolor{blue}{\url{http://www.faqs.org/faqs/unix-faq/shell/shell-differences/}}
        }
}

\fr{Sistemi Windows}{
  \bl{}{
    L'interprete comandi è rappresentato dal programma \texttt{cmd.exe} in \texttt{C:$\backslash$Windows$\backslash$System32$\backslash$cmd.exe}
    \iz{
      \item Eredita in realtà sintassi e funzionalità della maggior parte dei comandi del vecchio MSDOS
      \item Versioni recenti hanno introdotto PowerShell, basato su .NET e C\#
      \item Windows 10 ha introdotto il supporto a bash tramite Linux Subsystem for Windows
    }
  }
  \fg{height=0.4\textheight}{img/prompt.png}
}

\begin{frame}[fragile]{Aprire un terminale in laboratorio}
  \begin{itemize}
    \item In laboratorio, troverete il terminale (prompt dei comandi) clickando su Start $\Rightarrow$ Programmi $\Rightarrow$ Accessori $\Rightarrow$ Prompt dei comandi
    \item Metodo più rapido: Start  $\Rightarrow$ Nella barra di ricerca, digitare \texttt{cmd} $\Rightarrow$ clickare su \texttt{cmd.exe}
  \end{itemize}
\end{frame}

\begin{frame}[fragile]{File system e terminale: cheat sheet}
\label{slide:commands}
  \begin{center}
    \begin{tabular}{| l | c | c |}
      \hline
      \textbf{Operazione} & \textbf{Comando Unix} & \textbf{Comando Win} \\ \hline
      \scriptsize{}Visualizzare la directory corrente & \texttt{pwd} & \texttt{echo \%cd\%}  \\ \hline
      \scriptsize{}Eliminare il file \texttt{f} (non va con le cartelle!) & \texttt{rm f} & \texttt{del f} \\ \hline
      \scriptsize{}Eliminare la directory \texttt{nd} & \texttt{rm -r nd} & \texttt{rd nd} \\ \hline
      \scriptsize{}Contenuto della directory corrente & \texttt{ls -alh} & \texttt{dir} \\ \hline
%       \scriptsize{}Avviare un eseguibile di nome \texttt{pippo} & \texttt{./p} & \texttt{.\textbackslash{}p} \\ \hline
      \scriptsize{}Cambiare unità disco (passare a D:) & -- & \texttt{D:} \\ \hline
      \scriptsize{}Passare alla directory \texttt{nd} & \texttt{cd nd} & \texttt{cd nd} \\ \hline
      \scriptsize{}Passare alla directory di livello superiore & \texttt{cd ..} & \texttt{cd..} \\ \hline
      \scriptsize{}Spostare (rinominare) un file \texttt{f1} in \texttt{f2} & \texttt{mv f1 f2} & \texttt{move f1 f2} \\ \hline
      \scriptsize{}Copiare il file \texttt{f} in \texttt{fc} & \texttt{cp f fc} & \texttt{copy f fc} \\ \hline
      \scriptsize{}Creare la directory \texttt{d} & \texttt{mkdir d} & \texttt{md d} \\ \hline
    \end{tabular}
  \end{center}
%  Eseguire delle prove ed esser certi di aver compreso come utilizzare ogni comando. Per \emph{cominciare} l'esame, in particolare, dovrete usare il comando \texttt{cd}: siate certi di aver capito cosa fa!
\end{frame}

\begin{frame}[fragile]{Uso intelligente del terminale (1/2)}
  \begin{block}{Autocompletamento}
    \scriptsize{}
    Sia *nix che Windows offrono la possibilità di effettuare autocompletamento, ossia chiedere al sistema di provare a completare un comando. 
	\begin{itemize}
	\item Per farlo si utilizza il tasto ``tab''.
	\item Es. Se ci si trova in una direcotry contenenti i file \emph{Main.java} e \emph{Test.java}, digitando ``T'' e premendo poi il tasto ``tab'' verrà automaticamente completato il nome del file \emph{Test.java}
\end{itemize}	    
  \end{block}
  \vfill
  \begin{block}{Memoria dei comandi precendenti}
    \scriptsize{}
    Sia *nix che Windows offrono la possibilità di richiamare rapidamente i comandi inviati precedentemente premendo il tasto ``freccia su''. 
\begin{itemize}
\item  I sistemi *nix supportano anche il lancio di comandi eseguiti in sessioni precedenti (non perde memoria col riavvio del terminale). 
\end{itemize}    
   
  \end{block}

\end{frame}

\begin{frame}[fragile]{Uso intelligente del terminale (2/2)}
  \begin{block}{Interruzione di un programma}
    \scriptsize{}
    È possibile interrompere forzatamente un programma, qualora non risponda per vari motivi o non si voglia attenderne la normale terminazione. 

\begin{itemize}
\item Per farlo, sia su Windows che in *nix, si prema ctrl+c.
\end{itemize}    
  \end{block}
  \vfill
  \begin{block}{Ricerca nella storia dei comandi precedenti}
    \scriptsize{}
    \begin{itemize}
    \item Premendo ctrl+r seguito da un testo da cercare, i sistemi *nix supportano la ricerca all'interno dei comandi lanciati recentemente, anche in sessioni utente precedenti. Non disponibile su Windows.
    \item Su Windows è possibile navigare la history dei comandi lanciati premendo F7 (funzione limitata alla sola sessione in corso)
    \end{itemize}

  \end{block}
\end{frame}

\section{Preparazione dell'ambiente di lavoro (per Lab01)}

\fr{Preparazione Ambiente di Lavoro (1/4)}{
	\iz{
		\item Accendere il PC ed eseguire l'accesso
		\item Accedere al sito del corso
		\iz{
			\item \textcolor{blue}{\url{http://bit.ly/oop19-cesena}}
		}
		\item Scaricare dalla sezione dedicata a questa settimana il materiale dell'esercitazione odierna
		\item Spostare il file scaricato sul Desktop
		\item Decomprimere il file sul Desktop
	}
}


\fr{Preparazione Ambiente di Lavoro (2/4)}{
	\iz{
		\item Aprire il prompt dei comandi 
	}	
	\fg{height=0.5\textheight}{img/prompt01.jpg}
	
	\iz{
	\item NOTA: generalmente il prompt si apre nella directory principale dell'utente
	}
}

\fr{Preparazione Ambiente di Lavoro (3/4)}{
	\iz{
		\item Posizionarsi all'interno della cartella scompattata con l'ausilio del comando \texttt{cd} (change directory)
		\en{
			\item \texttt{cd Desktop} e premere invio, dopodiché
			\item \texttt{cd lab01} e premere invio
		}
	}
	\fg{height=0.6\textheight}{img/prompt02.jpg}
}

\begin{frame}{Preparazione Ambiente di Lavoro (4/4)}
\begin{itemize}
\item Scegliere un editor di testo per modificare i file sorgente
\begin{itemize}
\item Il laboratorio è equipaggiato con diversi editor di testo: JEdit, Notpad++, Atom
\item NOTA: \textbf{NON} sono adatti per l'apertura e la modifica di file Java (né di alcun altro linguaggio di programmazione) i word processors (Libreoffice Writer, Openoffice Writer, Microsoft Word, Microsoft WordPad...), né l'editor di testo incluso in Windows (Notepad).
\end{itemize}
\vfill
\item Si suggerisce l'utilizzo di \textbf{Atom} (\texttt{https://atom.io/})
\begin{itemize}
\item fornisce un supporto adeguato alla scrittura/modifica di sorgenti Java
\item offre la possibilità di interpretare la sintassi markdown: modalità con cui sono forniti i testi degli esercizi (per i primi laboratori)
\end{itemize}
\end{itemize}
\end{frame}

\begin{frame}{Preparazione Ambiente di Lavoro (4/4)}
\fg{width=\textwidth}{img/atom.jpg}
\end{frame}

\section{Compilazione ed Esecuzione da Riga di Comando}

\fr{Compilazione ed Esecuzione, comandi di base}{
	\bl{Compilazione}{
		Compilazione di una classe (comando \textcolor{blue}{javac})
		\iz{
			\item \alert{\cil{javac NomeFileSorgente.java}} 
			\item \alert{\cil{javac *.java}} compila tutti i sorgenti nella directory corrente
		}
	}
	\bl{Esecuzione}{
		Esecuzione di un programma Java (comando \textcolor{blue}{java})
		\iz{
			\item	\alert{\cil{java NomeDellaClasse}}
		}
	}
	Si noti che il compilatore \emph{traduce} \textbf{file} sorgente in \textbf{file} binari, 
mentre \\
    L'interprete Java (la Java Virtual Machine) esegue una ed una sola \textbf{classe}.
    
    Il compilatore Java lavora su \textit{file}, l'interprete Java su \textit{classi}
}

\section{Appendice : richiami utili per gli esercizi del Lab01}

\fr{A1 -- Operazioni con i Numeri Complessi}{
\bl{Numeri complessi -- breve ripasso}{
Siano $z,w \in \mathbb{C} : z = a + ib,\ w = c + id$, allora:
\iz {
\item Confronto: $z = w \iff a = c \wedge b = d$
\item Somma algebrica: $z \pm w = a \pm c + i(b \pm d)$
\item Prodotto: $z \cdot w = (a+ib)(c+id) = ac-bd+i(bc+ad)$
\item Rapporto: $\frac{z}{w} = \frac{(a+ib)(c-id)}{(c+id)(c-id)} = \frac{ac + bd}{c^2 + d^2} + i\frac{bc - ad}{c^2 + d^2}$
}
}	
}

\end{document}

\begin{frame}[allowframebreaks]
 \frametitle{Bibliography}
	\bibliographystyle{plain}
	\small
 \bibliography{biblio}
\end{frame}

