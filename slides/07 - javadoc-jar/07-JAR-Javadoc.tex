%\documentclass[12pt,handout]{beamer}
\documentclass[xcolor=dvipsnames,presentation]{beamer}
\usepackage{../oop-slides-lab}

\setbeamertemplate{bibliography item}[text]

\newcommand{\lab}{Lab07}

\title[{\lab} -- Jar, Javadoc]{JAR e Javadoc}

\date[\today]{\today}

\begin{document}

\frame[label=coverpage]{\titlepage}

\newcommand{\al}[0]{\textless}
\newcommand{\ar}[0]{\textgreater}
\newcommand{\gen}[1]{\al{}#1\ar{}}
\newcommand{\imgfr}[4]{\fr{#1}{#2
\begin{center}
\includegraphics[width=#3\textwidth]{#4}
\end{center}
}}

%====================
%Outline
%====================

\section{Deployment di applicazioni Java}
\subsection{I JAR file}
\fr{Deployment di applicazioni Java}{
  \iz {
    \item Finora, abbiamo visto che un'applicazione Java è composta di un insieme di classi: noi vorremmo distribuirla come singolo file!
    \item Normalmente, le applicazioni Java vengono confezionate in file JAR (Java ARchive)
  }
  \bl{File JAR} {
    \iz {
      \item Un JAR è un archivio ZIP che contiene le classi, le risorse (e.g. icone) e un file descrittivo detto ``Manifest''
      \item Il Manifest viene creato sempre in \texttt{META-INF/MANIFEST.MF}, e contiene informazioni sull'applicazione, ad esempio su quale classe contenga il main del programma
      \item È possibile associare (a livello di sistema operativo) l'esecuzione del JAR file al comando Java, in modo che l'applicazione si avvii automaticamente ``col doppio click'' (avviando automaticamente\ la classe scritta nel Manifest)
      \item È possibile utilizzare i file JAR come componenti di altre applicazioni
    }
  }
}

\fr{Il comando \texttt{jar}}{
    \iz{
        \item Per creare un file JAR, si può utilizzare direttamente il comando \texttt{jar}
        \iz{
            \item Normalmente, tuttavia, ci si avvale del supporto dell'IDE
            \item O meglio ancora, di strumenti di build automation (Gradle, Maven...)
            \begin{itemize}
                \item Non sono parte del corso...
            \end{itemize}
        }
    }
    \bl{Opzioni importanti del comando \texttt{jar}}{
    \textbf{\texttt{c}} --- Specifica l'intenzione di creare un JAR file\\
    \textbf{\texttt{f}} --- Specifica un file di output (se non presente, l'output è rediretto su standard output)\\
    \textbf{\texttt{m}} --- Specifica l'intenzione di allegare un manifest file personalizzato (se non presente, ne viene creato uno di default, che non specifica alcuna classe da eseguire)
}
}

\fr{Esempi d'uso del comando \texttt{jar}}{

\bl{\texttt{jar cf jar-file.jar file1 file2 directory1}}{
\iz {
\item Crea un nuovo JAR file di nome \texttt{jar-file.jar} contenente \texttt{file1} \texttt{file2} \texttt{directory1}. Include un Manifest di default.
}
}

\bl{\texttt{jar cf jar-file.jar *}}{
\iz {
\item Crea un nuovo JAR file di nome \texttt{jar-file.jar} contenente tutti i file e le directory nel path corrente. Include un Manifest di default.
}
}


\bl{\texttt{jar cfm jar-file.jar MYMANIFEST it/unibo/*}}{
\iz {
\item Crea un nuovo JAR file, di nome \texttt{jar-file.jar}, contenente tutti i file e le directory nel path \texttt{it/unibo}, con manifest dato dal file \texttt{MYMANIFEST} nella cartella corrente.
}
}

}


\fr{Esecuzione di JAR file tramite command line shell}{
  \iz {
    \item \texttt{java} ha un'opzione che consente l'esecuzione di file jar.
    \item Tale opzione è \texttt{-jar}
    \item Quando si lancia \texttt{java -jar nomefile.jar}, la Java Virtual Machine automaticamente legge il file Manifest, cerca una descrizione della Main Class da eseguire e tenta di eseguirla.
  }
}

\subsection{Generazione di JAR tramite Eclipse}
\fr {Generazione di file JAR con Eclipse (1/5)} {
\iz {
\item L'IDE Eclipse supporta efficacemente l'utente nella generazione di file JAR (eseguibili o per la creazione di librerie).
}
\bl{Esercizio}{
\iz {
\item Si seguano i passi descritti nelle slide successive per generare un file JAR a partire dai tre packages:
\iz{
\item \texttt{it.unibo.oop.jar.packages.pkg1}
\item \texttt{it.unibo.oop.jar.packages.pkg2}
\item \texttt{it.unibo.oop.jar.packages.pkg3}
}
}
}
}

\fr {Generazione di file JAR con Eclipse (2/5)}{
\iz{
\item Click col tasto destro sul progetto $>$ Export...
\item Nella schermata di export: java $>$ JAR file $>$ Next
}
\fg{height=0.8\textheight}{img/exportjar4}
}

\fr {Generazione di file JAR con Eclipse (3/5)}{
\iz{
\item Espandere l'elenco dei file: verificare \textbf{sempre} che siano selezionate \textbf{tutte e sole} le risorse d'interesse per il JAR file
}
\fg{height=0.75\textheight}{img/exportjar6}
}

\fr {Generazione di file JAR con Eclipse (4/5)}{
\iz{
\item Risulta possibile esportare anche i sorgenti nel file JAR
\iz{
\item Utile nel caso si voglia rendere ispezionabile il codice sorgente delle librerie distribuite come file JAR (ad esempio a fini di debug).
}
\item In questo caso, selezionare: \emph{Export Java Source files and resources}
}
\fg{height=0.65\textheight}{img/exportjar7}
}

\fr {Generazione di file JAR con Eclipse (5/5)}{
\iz{
\item Selezionare una directory di destinazione
\item Fornire un nome (con estensione .jar) al file che sarà generato
\item Click su Finish
}
\fg{height=0.7\textheight}{img/exportjar9}
}

%\imgfr{Generazione di file JAR con Eclipse}{Selezionare le risorse da esportare, click destro ed \texttt{Export}}{0.3}{img/exportjar1}
%\imgfr{Generazione di file JAR con Eclipse}{}{0.6}{img/exportjar2}
%\imgfr{Generazione di file JAR con Eclipse}{}{0.6}{img/exportjar3}
%\imgfr{Generazione di file JAR con Eclipse}{}{0.6}{img/exportjar4}
%\imgfr{Generazione di file JAR con Eclipse}{Espandere l'elenco dei file}{0.4}{img/exportjar5}
%\imgfr{Generazione di file JAR con Eclipse}{Verificare \textbf{sempre} che siano selezionate \textbf{tutte e sole} le risorse che ci interessano.}{0.5}{img/exportjar6}
%\imgfr{Generazione di file JAR con Eclipse}{Eventualmente potete esportare anche i sorgenti}{0.5}{img/exportjar7}
%\imgfr{Generazione di file JAR con Eclipse}{Inserite il nome del file (e selezionate la cartella dove volete creare il JAR)}{0.5}{img/exportjar8}
%\imgfr{Generazione di file JAR con Eclipse}{Finish}{0.5}{img/exportjar9}

\fr {Verifica del contenuto di un JAR} {
    \iz{
        \item Il tool \texttt{jar} distribuito con il JDK consente di verificare il contenuto del file compresso
    \iz{
        \item Aprire un terminale
        \item Digitare \texttt{jar tf myjar.jar} (dove \texttt{myjar.jar} è il nome del file JAR da verificare
    }
    }
}

\subsection{Generazione di JAR eseguibili tramite Eclipse}

\fr {Generazione di file JAR eseguibili con Eclipse (1/3)} {
    \iz {
        \item Eclipse consente anche di generare file JAR con un file Manifest ad hoc per mettere in esecuzione il progetto software contenuto nell'archivio.
        \item Come prerequisito, è necessario che esista in Eclipse una ``Run Configuration'' valida per il progetto.
        \iz{
            \item Ad esempio se il progetto sia stato eseguito almeno una volta in Eclipse.
        }
    }
}

\fr {Generazione di file JAR eseguibili con Eclipse (2/3)}{
\iz{
\item Clik DX su progetto $>$ Export...
\item Nella schermata di export: java $>$ Runnable JAR file $>$ Next
}
\fg{height=0.78\textheight}{img/exportrunnablejar6}
}

\fr {Generazione di file JAR eseguibili con Eclipse (3/3)}{
\iz{
\item Selezionare la ``Run configuration'' da utilizzare
\item Inserire il nome del file (e selezionare il path in cui si desidera crearlo)
}
\fg{height=0.78\textheight}{img/exportrunnablejar9}
}

\section{Documentazione per il codice Java}

\subsection{Javadoc}

\fr{Documentazione per il Codice Sorgente}{
  \bl{Il ruolo fondamentale della documentazione}{
    La documentazione di un progetto software è un aspetto fondamentale
    \iz{
      \item Al fine di garantirne la manutenibilità
        \iz{
          \item Siamo così sicuri di ricordarci a distanza di settimane del perché abbiamo adottato una certa architettura, scritto una certa classe, una data estensione di una classe, un certo metodo?
          \item Arriva in azienda un nuovo sviluppatore: come fa a costruirsi il background necessario per lavorare su un progetto software esistente?
        }
      \item Al fine di aumentare la comprensione del codice
        \iz{
          \item Cosa farà mai il metodo \cil{doStuff()} sviluppato dal collega?
          \item Ci sono dei metodi che sono disponibili solo per ragioni di compatibilità e che non dovrei più utilizzare?
        }
    }
  }

  \bl{}{
    Oggi ci concentreremo sulla \alert{Javadoc} che, integrata all'uso di UML, è una delle parti fondamentali di una buona documentazione di un progetto SW
  }
}

\fr{Javadoc -- Introduzione}{
  \bl{Javadoc: che cos'è?}{
    Tool di supporto per la generazione automatica della documentazione (HTML-based) dei programmi Java, tramite utilizzo di annotazioni \textit{``speciali''} collocate in punti ben precisi dei sorgenti
  }

  \bl{Come funziona}{
    \iz{
      \item Il tool processa tutti i commenti del tipo \cil{/** ... */}
      \item Commenti che si trovano in testa a dichiarazione di classi, metodi, etc. vengono inclusi per generare automaticamente la documentazione
      \item Custom tag consentono di classificare diverse tipologie di informazioni, facilitando la scrittura e la generazione della documentazione
    }
  }
}

\subsection{Principali Tag e Linee Guida}

\fr{Javadoc -- Principali Tag (1/2)}{
  \iz{
    \item \textcolor{blue}{\cil{@param}}
      \iz{
        \item Utilizzabile nei commenti relativi a costruttori/metodi o in classi parametriche
        \item Descrive un generico parametro di input oppure un tipo generico
      }
  }
  \iz{
    \item \textcolor{blue}{\cil{@return}}
      \iz{
        \item Utilizzabile nei commenti relativi ai metodi
        \item Descrive il valore di ritorno
      }
  }
  \iz{
    \item \textcolor{blue}{\cil{@deprecated}}
      \iz{
        \item Utilizzabile nei commenti relativi a classi/costruttori/metodi/campi
        \item Descrive che quel particolare costruttore/metodo etc. è stato deprecato. E' ancora disponibile per ragioni di compatibilità ma è opportuno non utilizzarlo nello sviluppo di nuove applicazioni
      }
  }
}

\fr{Javadoc -- Principali Tag (2/2)}{
  \iz{
    \item \textcolor{blue}{\cil{@throws}}
      \iz{
        \item Utilizzabile nei commenti relativi a costruttori/metodi che lanciano una eccezione (le vedremo in seguito)
        \item Descrive l'eccezione e il motivo per cui viene lanciata
      }
  }

  \iz{
    \item \textcolor{blue}{\cil{@link}}
      \iz{
        \item Descrive collegamenti ipertestuali (link) a metodi/classi/costanti della stessa classe o anche di classi esterne
        \item Esempi di utilizzo
          \iz{
            \item \texttt{ \{@link package.OtherClass\#someMethod\} }
            \item \texttt{ \{@link \#someMethodOfSameClass\} }
            \item \texttt{ \{@link \#someFieldOfSameClass\} }
          }
      }
  }

  \iz{
    \item \textcolor{blue}{\cil{@author}} ed \textcolor{blue}{\cil{@version}}
      \iz{
        \item Utilizzabili nel commento che descrive la classe
        \item Descrive l'autore/autori della classe, e la sua versione
        \item \textcolor{red}{Non usateli}. Abbiamo il DVCS per sapere esattamente chi ha fatto
cosa, quando, e qual è la versione della classe.
      }
  }
}

\fr{Javadoc -- Linee Guida}{
  \bl{Cosa commentare e quali tag usare?}{
    \iz{
      \item Inserire sempre un commento (anche corto) che descrive il ruolo e il funzionamento generale della classe
      \item Inserire un commento per tutti i costruttori (parametri e return value inclusi), metodi, e campi con livello di accesso \cil{public} e \cil{protected}
      \item Non è necessario documentare metodi di cui si fa override, a meno che non vi siano peculiarità non rilevate nella documentazione della superclasse. In questo caso, è sufficiente utilizzare \textcolor{red}{\texttt{\{@inheritDoc\}}} all'interno del commento, e verrà generata Javadoc prendendola dalla superclasse.
    }
  }
  \bl{}{
    Utilizzare i sorgenti che vi forniamo in lab come linea guida di riferimento
  }
}

\subsection{Generazione della Javadoc tramite Eclipse}

\fr{Generazione Javadoc tramite Eclipse (1/2)}{
\fg{height=0.9\textheight}{img/javadoc1}
}

\fr{Generazione Javadoc tramite Eclipse (2/2)}{
\fg{height=0.8\textheight}{img/javadoc2}
}

\fr{Javadoc: Esempio di Documentazione Generata}{
\fg{height=0.8\textheight}{img/javadoc_view}
}

\section{Preparazione al laboratorio}

\begin{frame}{Modalità di lavoro}
    \begin{enumerate}
        \item Nella root del repository c'è un file README.md che descrive le operazioni da svolgere
        \item Tentare di capire l'esercizio in autonomia
        \begin{itemize}
            \item Contattare il docente se qualcosa non è chiaro
        \end{itemize}
        \item Risolvere l'esercizio in autonomia
        \begin{itemize}
            \item Contattare il docente se si rimane bloccati
        \end{itemize}
        \item \textcolor{red}{Utilizzare le funzioni di test presenti nei sorgenti per il testing dell'esercizio}
        \item Cercare di risolvere autonomamente eventuali piccoli problemi che possono verificarsi durante lo svolgimento degli esercizi
        \begin{itemize}
            \item Contattare il docente se, anche dopo aver usato il \textbf{debugger}, non si è riusciti a risalire all'origine del problema
        \end{itemize}
        \item \textcolor{red}{Scrivere la documentazione Javadoc per l'esercizio svolto}
        \item Effettuare almeno un commit ad esercizio completato
        \begin{itemize}
            \item Fatene pure quanti ne volete durante l'esercizio stesso
        \end{itemize}
        \item \textbf{A esercizio ultimato sottoporre la soluzione al docente}
        \item Proseguire con l'esercizio seguente
    \end{enumerate}
\end{frame}

\end{document}
