\documentclass[presentation]{beamer}
\usepackage{../oop-slides-pianini}
\usepackage{url}
\setbeamertemplate{bibliography item}[text]
\title[OOP04 -- Ereditarietà]{04 \\Uso avanzato di Eclipse, ereditarietà e classi astratte}
\author[Pianini]{Danilo Pianini \\ Giovanni Ciatto, Angelo Croatti, Mirko Viroli}

\begin{document}

\frame[label=coverpage]{\titlepage}

\section{Lab Startup}

\fr{Preparazione Ambiente di Lavoro 1/2}{
  \iz{
    \item Accendere il PC
    \item Loggarsi sul sito del corso
    \iz{
      \item\textcolor{blue}{\url{https://bit.ly/oop2016cesena}}
    }
    \item Scaricare dalla sezione \texttt{lab} del sito il file \texttt{lab04.zip} contenente il materiale dell'esercitazione odierna
    \item Spostare il file scaricato sul Desktop
    \item Decomprimere il file usando 7zip (o un programma analogo) sul Desktop
    \item Copiare la cartella scompattata (\texttt{lab04}) nel vostro workspace di Eclipse
    \iz{
      \item p.e. \texttt{C:$\backslash$Users$\backslash$<username>$\backslash$workspace}
    }
    \item Importare il progetto \texttt{lab04} con la procedura di importazione dei progetti vista durante lo scorso laboratorio
  }
  
}


\fr{Modalità di Lavoro}{
  \bl{}{
    \en{
      \item Leggere la consegna
      \item Risolvere l'esercizio in autonomia
      \item Cercare di risolvere autonomamente eventuali piccoli problemi che possono verificarsi durante lo svolgimento degli esercizi
      \item \textcolor{red}{Utilizzare le funzioni di test presenti nei sorgenti per il testing dell'esercizio}
      \item Contattare i docenti nel caso vi troviate a lungo bloccati nella risoluzione di uno specifico esercizio
      \item A esercizio ultimato contattare i docenti per un rapido controllo della soluzione realizzata
%       \item \textcolor{red}{Scrivere la Javadoc per l'esercizio svolto} (o in lab o a casa in mancanza di tempo)
      \item Proseguire con l'esercizio seguente
    }
  }
}

\section{Consigli per l'esercitazione odierna}

\fr{Design di una applicazione}{
  \iz {
    \item Come ingegneri, dovrete essere in grado di tradurre una descrizione in linguaggio naturale (``a parole'') di una applicazione in termini di astrazioni adatte ad essere scritte in forma di software (nel nostro caso, in elementi di linguaggio di programmazione orientato agli oggetti)
    \item Si tratta dell'esercizio \textbf{più difficile} e \textbf{più importante} che dovrete fare quando realizzerete il progetto
    \item Dalle vostre capacità di analisi del problema e di design della soluzione dipende il successo o meno del vostro software
    \item Nel prossimo esercizio, vi sarà data una descrizione in linguaggio naturale, e starà a voi realizzare il software autonomamente
    \item \textbf{SUGGERIMENTO} --- Prima di iniziare a ``sporcarsi le mani'' col codice, prendete carta e penna e cercate di ottenere un design chiaro dei componenti che realizzerete: nessuno costruisce un'automobile partendo con l'assemblare i bulloni, ma decide prima quali saranno i componenti e come andranno connessi fra loro.
  }
}

\end{document}

