\documentclass[presentation]{beamer}
\usepackage{../oop-slides-pianini}
\setbeamertemplate{bibliography item}[text]
\newcommand{\lessonnr}[0]{05}
\title[OOP05 -- Collections and Generics I]{05 \\JAR file, Javadoc, Polimorfismo Parametrico e Introduzione al Java Collections Framework}

\begin{document}
	
\frame[label=coverpage]{\titlepage}

\newcommand{\al}[0]{\textless}
\newcommand{\ar}[0]{\textgreater}
\newcommand{\gen}[1]{\al{}#1\ar{}}
\newcommand{\imgfr}[4]{\fr{#1}{#2
\begin{center}
\includegraphics[width=#3\textwidth]{#4}                    
\end{center}
}}

%====================
%Outline
%====================

\fr{Outline}{
\bl{Goal della lezione}{
\iz{ 
\item Introduzione ai Java ARchive (JAR) e loro utilizzo
\item Il ruolo della documentazione nello sviluppo del software
\iz{
\item Javadoc: linee guida
\item Generazione automatica
}
\item Esercizi in Autonomia
\iz{
\item Polimorfismo Parametrico
\item Java Collection Framework - Parte I
}
}
}
}


\fr{Preparazione Ambiente di Lavoro}{
  \iz{
    \item Accendere il PC
    \item Loggarsi sul sito del corso
    \iz{
      \item\textcolor{blue}{\url{https://bit.ly/oop2016cesena}}
    }
    \item Scaricare dal sito il file \texttt{lab05.zip} contenente il materiale dell'esercitazione odierna
    \item Spostare il file scaricato sul Desktop
    \item Decomprimere il file usando 7zip (o un programma analogo) sul Desktop
    \item Importare la cartella \texttt{lab05} come progetto all'interno di Eclipse
  }
}


\section{Deployment di applicazioni Java}
\subsection{I JAR file}
\fr{Deployment di applicazioni Java}{
  \iz {
    \item Finora, abbiamo visto che un'applicazione Java è composta di un insieme di classi: noi vorremmo distribuirla come singolo file!
    \item Normalmente, le applicazioni Java vengono confezionate in file JAR (Java ARchive)
  }
  \bl{File JAR} {
    \iz {
      \item Un JAR è un archivio (normalmente compresso) che contiene le classi, le risorse (e.g. icone) e un file descrittivo detto ``Manifest''
      \item Il Manifest viene creato sempre in \texttt{META-INF/MANIFEST.MF}, e contiene informazioni sull'applicazione, ad esempio su quale classe contenga il main del programma
      \item È possibile associare (a livello di sistema operativo) l'esecuzione del JAR file al comando Java, in modo che l'applicazione si avvii automaticamente ``col doppio click'' (avviando automaticamente\ la classe scritta nel Manifest)
      \item È possibile utilizzare i file JAR come componenti di altre applicazioni
    }
  }
}

\fr{Il comando \texttt{jar}}{
\iz{
\item Per creare un file JAR, si può utilizzare direttamente il comando \texttt{jar}
\iz{
\item Normalmente, tuttavia, ci si avvale del supporto dell'IDE
}
}
\bl{Opzioni importanti del comando \texttt{jar}}{
\textbf{\texttt{c}} --- Specifica l'intenzione di creare un JAR file\\
\textbf{\texttt{f}} --- Specifica un file di output (se non presente, l'output è rediretto su standard output)\\
\textbf{\texttt{m}} --- Specifica l'intenzione di allegare un manifest file personalizzato (se non presente, ne viene creato uno di default, che non specifica alcuna classe da eseguire)
}
}

\fr{Esempi d'uso del comando \texttt{jar}}{

\bl{\texttt{jar cf jar-file.jar file1 file2 directory1}}{
\iz {
\item Crea un nuovo JAR file di nome \texttt{jar-file.jar} contenente \texttt{file1} \texttt{file2} \texttt{directory1}. Include un Manifest di default.
}
}

\bl{\texttt{jar cf jar-file.jar *}}{
\iz {
\item Crea un nuovo JAR file di nome \texttt{jar-file.jar} contenente tutti i file e le directory nel path corrente. Include un Manifest di default.
}
}
}


\fr{Esecuzione di JAR file tramite command line shell}{
  \iz {
    \item \texttt{java} ha un'opzione che consente l'esecuzione di file jar.
    \item Tale opzione è \texttt{-jar}
    \item Quando si lancia \texttt{java -jar nomefile.jar}, la Java Virtual Machine automaticamente legge il file Manifest, cerca una descrizione della Main Class da eseguire e tenta di eseguirla.
  }
}

\subsection{Generazione di JAR tramite Eclipse}
\fr {Generazione di file JAR con Eclipse (1/5)} {
\iz {
\item L'IDE Eclipse supporta efficacemente l'utente nella generazione di file JAR (eseguibili o per la creazione di librerie).
}
\bl{Esercizio}{
\iz {
\item Si seguano i passi descritti nelle slide successive per generare un file JAR a partire dai tre packages:
\iz{
\item \texttt{it.unibo.oop.lab05.packages.pkg1} 
\item \texttt{it.unibo.oop.lab05.packages.pkg2}
\item \texttt{it.unibo.oop.lab05.packages.pkg3}
}
}
}
}

\fr {Generazione di file JAR con Eclipse (2/5)}{
\iz{
\item Click col tasto destro sul progetto $>$ Export...
\item Nella schermata di export: java $>$ JAR file $>$ Next
}
\fg{height=0.8\textheight}{img/exportjar4} 
}

\fr {Generazione di file JAR con Eclipse (3/5)}{
\iz{
\item Espandere l'elenco dei file: verificare \textbf{sempre} che siano selezionate \textbf{tutte e sole} le risorse d'interesse per il JAR file
}
\fg{height=0.75\textheight}{img/exportjar6} 
}

\fr {Generazione di file JAR con Eclipse (4/5)}{
\iz{
\item Risulta possibile esportare anche i sorgenti nel file JAR
\iz{
\item Utile nel caso si voglia rendere ispezionabile il codice sorgente delle librerie distribuite come file JAR (ad esempio) in fase di debug.
}
\item In questo caso, selezionare: \emph{Export Java Source files and resources}
}
\fg{height=\textheight}{img/exportjar7} 
}

\fr {Generazione di file JAR con Eclipse (5/5)}{
\iz{
\item Selezionare una directory di destinazione
\item Fornire un nome (con estensione .jar) al file che sarà generato
\item Click su Finish
}
\fg{height=0.75\textheight}{img/exportjar9} 
}

%\imgfr{Generazione di file JAR con Eclipse}{Selezionare le risorse da esportare, click destro ed \texttt{Export}}{0.3}{img/exportjar1}
%\imgfr{Generazione di file JAR con Eclipse}{}{0.6}{img/exportjar2}
%\imgfr{Generazione di file JAR con Eclipse}{}{0.6}{img/exportjar3}
%\imgfr{Generazione di file JAR con Eclipse}{}{0.6}{img/exportjar4}
%\imgfr{Generazione di file JAR con Eclipse}{Espandere l'elenco dei file}{0.4}{img/exportjar5}
%\imgfr{Generazione di file JAR con Eclipse}{Verificare \textbf{sempre} che siano selezionate \textbf{tutte e sole} le risorse che ci interessano.}{0.5}{img/exportjar6}
%\imgfr{Generazione di file JAR con Eclipse}{Eventualmente potete esportare anche i sorgenti}{0.5}{img/exportjar7}
%\imgfr{Generazione di file JAR con Eclipse}{Inserite il nome del file (e selezionate la cartella dove volete creare il JAR)}{0.5}{img/exportjar8}
%\imgfr{Generazione di file JAR con Eclipse}{Finish}{0.5}{img/exportjar9}

\fr {Verifica del contenuto di un JAR} {
	\iz{
		\item Il tool \texttt{jar} distribuito con il JDK consente di verificare il contenuto del file compresso
	\iz{
		\item Aprire un terminale
		\item Digitare \texttt{jar tf myjar.jar} (dove \texttt{myjar.jar} è il nome del file JAR da verificare
	}
	}
}

\subsection{Generazione di JAR eseguibili tramite Eclipse}

\fr {Generazione di file JAR eseguibili con Eclipse (1/3)} {
\iz {
\item Eclipse consente anche di generare file JAR con un file Manifest ad hoc per mettere in esecuzione il progetto software contenuto nell'archivio.

\item Come prerequisito, è necessario che esista in Eclipse una ``Run Configuration'' valida per il progetto.
\iz{
\item Ovvero, è necessario che il progetto sia stato eseguito almeno una volta in Eclipse.
}
}
}

\fr {Generazione di file JAR eseguibili con Eclipse (2/3)}{
\iz{
\item Clik DX su progetto $>$ Export...
\item Nella schermata di export: java $>$ Runnable JAR file $>$ Next
}
\fg{height=0.8\textheight}{img/exportrunnablejar6} 
}

\fr {Generazione di file JAR eseguibili con Eclipse (3/3)}{
\iz{
\item Selezionare la ``Run configuration'' da utilizzare
\item Inserire il nome del file (e selezionare il path in cui si desidera crearlo)
}
\fg{height=0.8\textheight}{img/exportrunnablejar9} 
}

%\imgfr{Generazione di file JAR eseguibili con Eclipse}{}{0.99}{img/exportrunnablejar1}
%\imgfr{Generazione di file JAR eseguibili con Eclipse}{}{0.3}{img/exportrunnablejar2}
%\imgfr{Generazione di file JAR eseguibili con Eclipse}{}{0.4}{img/exportrunnablejar3}
%\imgfr{Generazione di file JAR eseguibili con Eclipse}{Selezionare le risorse di interesse}{0.99}{img/exportrunnablejar4}
%\imgfr{Generazione di file JAR eseguibili con Eclipse}{}{0.3}{img/exportrunnablejar5}
%\imgfr{Generazione di file JAR eseguibili con Eclipse}{}{0.7}{img/exportrunnablejar6}
%\imgfr{Generazione di file JAR eseguibili con Eclipse}{Selezionare la ``Run configuration'' da utilizzare}{0.7}{img/exportrunnablejar7}
%\imgfr{Generazione di file JAR eseguibili con Eclipse}{Inserire il nome del file (e selezionare il path in cui si desidera crearlo)}{0.7}{img/exportrunnablejar8}
%\imgfr{Generazione di file JAR eseguibili con Eclipse}{Finish}{0.7}{img/exportrunnablejar9}


\section{Documentazione per il codice Java}

\fr{Documentazione per il Codice Sorgente}{
  \bl{Il ruolo fondamentale della documentazione}{
    La documentazione di un progetto software è un aspetto fondamentale
    \iz{
      \item Al fine di garantirne la manutenibilità
        \iz{
          \item Siamo così sicuri di ricordarci a distanza di settimane del perché abbiamo adottato una certa architettura, scritto una certa classe, una data estensione di una classe, un certo metodo?
          \item Arriva in azienda un nuovo sviluppatore: come fa a costruirsi il background necessario per lavorare su un progetto software esistente?
        }
      \item Al fine di aumentare la comprensione del codice
        \iz{
          \item Cosa farà mai il metodo \cil{do_stuff()} sviluppato dal collega?
          \item Ci sono dei metodi che sono disponibili solo per ragioni di compatibilità e che non dovrei più utilizzare?
        }
    }
  }
                                  
  \bl{}{
    Oggi ci concentreremo sulla \alert{Javadoc} che, integrata all'uso di UML, è una delle parti fondamentali di una buona documentazione di un progetto SW
  }
}

\subsection{Javadoc}

\fr{Javadoc -- Introduzione}{
  \bl{Javadoc: che cos'è?}{
    Tool di supporto per la generazione automatica della documentazione (HTML-based) dei programmi Java, tramite utilizzo di annotazioni \textit{``speciali''} collocate in punti ben precisi dei sorgenti
  }
  
  \bl{Come funziona}{
    \iz{
      \item Il tool processa tutti i commenti del tipo \cil{/** ... */}
      \item Commenti che si trovano in testa a dichiarazione di classi, metodi, etc. vengono inclusi per generare automaticamente la documentazione
      \item Custom tag consentono di classificare diverse tipologie di informazioni, facilitando la scrittura e la generazione della documentazione
    }
  }
}

\subsection{Principali Tag e Linee Guida}

\fr{Javadoc -- Principali Tag (1/2)}{
  \iz{ 
    \item \textcolor{blue}{\cil{@param}}
      \iz{
        \item Utilizzabile nei commenti relativi a costruttori/metodi o in classi parametriche
        \item Descrive un generico parametro di input oppure un tipo generico
      }
  }
  \iz{ 
    \item \textcolor{blue}{\cil{@return}}
      \iz{
        \item Utilizzabile nei commenti relativi ai metodi
        \item Descrive il valore di ritorno
      }
  }
  \iz{ 
    \item \textcolor{blue}{\cil{@deprecated}}
      \iz{
        \item Utilizzabile nei commenti relativi a classi/costruttori/metodi/campi
        \item Descrive che quel particolare costruttore/metodo etc. è stato deprecato. E' ancora disponibile per ragioni di compatibilità ma è opportuno non utilizzarlo nello sviluppo di nuove applicazioni
      }
  }
}

\fr{Javadoc -- Principali Tag (2/2)}{
  \iz{ 
    \item \textcolor{blue}{\cil{@throws}}
      \iz{
        \item Utilizzabile nei commenti relativi a costruttori/metodi che lanciano una eccezione (le vedremo in seguito)
        \item Descrive l'eccezione e il motivo per cui viene lanciata
      }
  }
  
  \iz{ 
    \item \textcolor{blue}{\cil{@link}}
      \iz{
        \item Descrive collegamenti ipertestuali (link) a metodi/classi/costanti della stessa classe o anche di classi esterne
        \item Esempi di utilizzo
          \iz{
            \item \texttt{ \{@link package.OtherClass\#someMethod\} }
            \item \texttt{ \{@link \#someMethodOfSameClass\} }
            \item \texttt{ \{@link \#someFieldOfSameClass\} }
          }
      }
  }    

  \iz{ 
    \item \textcolor{blue}{\cil{@author}} ed \textcolor{blue}{\cil{@version}}
      \iz{
        \item Utilizzabili nel commento che descrive la classe
        \item Descrive l'autore/autori della classe, e la sua versione
        \item \textcolor{red}{Non usateli}. Esistono modi migliori (che vedremo in futuro in questo corso) per sapere esattamente chi ha fatto cosa, quando, e qual è la versione della classe.
      }
  }
}

\fr{Javadoc -- Linee Guida}{
  \bl{Cosa commentare e quali tag usare?}{
    \iz{
      \item Inserire sempre un commento (anche corto) che descrive il ruolo e il funzionamento generale della classe
      \item Inserire un commento per tutti i costruttori (parametri e return value inclusi), metodi, e campi con livello di accesso \cil{public} e \cil{protected}
      \item Non è necessario documentare metodi di cui si fa override, a meno che non vi siano peculiarità non rilevate nella documentazione della superclasse. In questo caso, è sufficiente utilizzare \textcolor{red}{\texttt{\{@inheritDoc\}}} all'interno del commento, e verrà generata Javadoc prendendola dalla superclasse.
    }
  }
  \bl{}{
    Utilizzare i sorgenti che vi forniamo in lab come linea guida di riferimento
  }
}

% \fr{Generazione Javadoc da Riga di comando}{
%   Esempio base di generazione della documentazione\\
%   
%   \iz{ 
%     \item {\tiny\texttt{javadoc -d <dest-dir> -sourcepath <src-path> -subpackages <base-pckg> -exclude <ex-pckg> -author}} 
%       \iz{
%         \item \textcolor{blue}{-d} directory in cui viene salvata la doc generata
%         \item \textcolor{blue}{-sourcepath} il path in cui trovare i sorgenti (da la possibilità di invocare il comando da qualunque cartella)
%         \item \textcolor{blue}{-subpackages} lista di package root da usare come base per generare la documentazione. Il comando andrà in ricorsione in tutti i sotto package
%         \item \textcolor{blue}{-exclude} lista di sotto package da escludere
%         \item \textcolor{blue}{-author} include il tag author alla doc generata
%       }
%   }
%   
%   Generiamo la documentazione per il progetto OOP1415-LAB (supponendo di trovarci in OOP1415-LAB)
%     \iz{
%       \item {\small\texttt{javadoc -d doc -sourcepath src -subpackages oop1415 -author}} 
%     }  
% }

\subsection{Generazione della Javadoc tramite Eclipse}

\fr{Generazione Javadoc tramite Eclipse (1/2)}{
\fg{height=0.9\textheight}{img/javadoc1} 
}

\fr{Generazione Javadoc tramite Eclipse (2/2)}{
\fg{height=0.8\textheight}{img/javadoc2}
}

\fr{Javadoc: Esempio di Documentazione Generata}{
\fg{height=0.8\textheight}{img/javadoc_view}
}

\section{Esercizi in Autonomia}

\fr{Modalità di Lavoro}{
  \bl{}{
    \en{
      \item Gli esercizi sono divisi in package con nomi progressivi
      \item Troverete un commento con le istruzioni per ciascun esercizio
      \item Risolvere l'esercizio in autonomia
      \item Cercare di risolvere autonomamente eventuali piccoli problemi che possono verificarsi durante lo svolgimento degli esercizi
      \item \textcolor{red}{Utilizzare le funzioni di test presenti nei sorgenti per il testing dell'esercizio}
      \item Contattare i docenti nel caso vi troviate a lungo bloccati nella risoluzione di uno specifico esercizio
      \item \textbf{Una volta ultimato l'esercizio chiamare i docenti per un controllo della soluzione}
       \item \textcolor{red}{Scrivere il Javadoc per l'esercizio svolto} 
      \item Proseguire con l'esercizio seguente
    }
  }
}

\begin{frame}{[Esercizio 1] \texttt{it.unibo.oop.lab05.ex1}}
\begin{enumerate}
\item Analizzare ed eseguire il programma definito in \texttt{UseCollection.java}
\item Seguire i commenti presenti in \texttt{UseSet.java} per realizzare un programma che crei, popoli e manipoli un oggetto di tipo \texttt{TreeSet<String>}
\end{enumerate}
\end{frame}

\begin{frame}{[Esercizio 1] Note Importanti (1/2)}
\begin{itemize}
\item Gli elementi inseriti in un \texttt{TreeSet} sono ordinati secondo l'\emph{ordine naturale}, ovvero il tipo specificato per gli oggetti contenuti nel \texttt{TreeSet} deve implementare l'interfaccia \texttt{Comparable}. Quindi, ad esempio:
\begin{itemize}
\item un set di tipo \texttt{TreeSet<String>} consente di aggiungere elementi al set attraverso il metodo \texttt{add} in quanto il tipo \texttt{String} implementa l'interfaccia \texttt{Comparable}
\item viceversa, se si tenta di aggiungere un qualunque elemento ad un set di tipo \texttt{TreeSet<MyClass>} -- dove \texttt{MyClass} rappresenta una classe che NON implementa l'interfaccia \texttt{Comparable} -- sarà generato un errore run-time di tipo \texttt{CastClassException} (i dettagli sulle eccezioni saranno descritti in una prossima lezione)
\end{itemize}
\end{itemize}
\end{frame}

\begin{frame}{[Esercizio 1] Note Importanti (2/2)}
\begin{itemize}
\item Tutti gli iteratori creati/richiesti per istanze di oggetti della classe \texttt{TreeSet} sono detti \emph{fail-fast}:
\begin{itemize}
\item Se il l'istanza di \texttt{TreeSet} è sottoposta a modifica dopo la creazione dell'iteratore (prima della rimozione dell'iteratore stesso), l'iteratore produrrà un errore run-time (nello specifico, sarà lanciata l'eccezione \texttt{ConcurrentModificationException} --- i dettagli sulle eccezioni saranno descritti in una prossima lezione)
\item \textbf{Provare a scrivere un ciclo foreach che itera sull'istanza di \texttt{TreeSet} -- che quindi crea internamente un iteratore -- tentando di eseguire sull'istanza il metodo \texttt{remove} all'interno del ciclo.} Funziona? No, perché si sta tentando di modificare il set dopo aver creato un iteratore sulla stessa istanza.
\end{itemize}
\end{itemize}

\end{frame}

\begin{frame}{[Esercizio 2] \texttt{it.unibo.oop.lab05.ex2}}
\begin{enumerate}
\item Seguire i commenti presenti in \texttt{UseSetWithOrder.java} per realizzare un programma che crei e ordini un oggetto di tipo \texttt{TreeSet<String>} avvalendosi di un oggetto da creare ad hoc che sia istanza di \texttt{Comparator<String>}
\item Si faccia riferimento alla documentazione di Java per verificare come poter utilizzare un Comparator per una specifica istanza di un TreeSet.
\end{enumerate}
\end{frame}

\begin{frame}{[Esercizio 3] \texttt{it.unibo.oop.lab05.ex3}}
\begin{enumerate}
\item Implementare le classi \texttt{WarehouseImpl.java} e \texttt{ProductImpl.java} secondo i contratti definiti nelle rispettive interfacce.
\item Si faccia riferimento alla Javadoc inserita per descrive le interfacce per i dettagli circa l'implementazione dei diversi metodi.
\item Si esegua la classe \texttt{UseWarehouse.java} e si verifichi il corretto funzionamento del programma.
\end{enumerate}
\end{frame}

\begin{frame}{[Esercizio 4] \texttt{it.unibo.oop.lab05.ex4}}
\begin{enumerate}
\item L'esercizio è una variazione di quanto fatto nell'esercizio precedente.
\item Si riutilizzi quanto più possibile il codice (funzionante) prodotto per completare l'esercizio 3
\item Per testare il buon funzionamento del codice prodotto, si completino -- nel metodo \texttt{main} della classe presente in \texttt{UseWarehouse.java} -- le inizializzazioni delle variabili di tipo \texttt{Product} e \texttt{Warehouse}.
\end{enumerate}
\end{frame}

\begin{frame}{[Esercizio 5] \texttt{it.unibo.oop.lab05.ex5}}
\begin{enumerate}
\item Si implementino le funzioni descritte nella classe \texttt{Utilities.java}
\item Si verifichi il buon funzionamento di quanto implementato eseguendo il programma descritto in \texttt{UseUtilities.java}
\end{enumerate}
\end{frame}


%\section{Java collections framework e Java generics}
%
%\fr {Istruzioni}{
%  \iz {
%    \item Nell'esercitazione di oggi troverete sorgenti parzialmente implementati
%    \item Direttamente dentro i sorgenti, troverete una serie di punti da svolgere
%    \item Per aiutarvi a trovarli, essi sono segnalati con il commento \cil{//TODO} ed un numero.
%    \item Eseguite gli esercizi tenendo conto della numerazione
%  }
%}
%
%%\imgfr{Struttura delle classi del model}{UML del modello}{0.99}{img/class}
%
%\subsection{Completamento di classi che usano collections}
%
%\fr {Utilizzo di Set\gen{E}}{
%  \iz {
%    \item Si completi la classe \texttt{oop1314.lab05.exams.model.Exam}.
%    \item È richiesto l'uso dell'interfaccia Set, come implementazione è consigliato HashSet.
%    \item Per dettagli sui metodi forniti dall'interfaccia, si consulti la Javadoc delle API Java
%  }
%}
%
%
%\subsection{Costruzione di una classe generica}
%
%\fr {Utilizzo di List\gen{E}: creazione di una memoria associativa}{
%  \iz {
%    \item Si osservi l'interfaccia \cil{oop1415.lab05.exams.generics.IAssociativeMemory<K, V>}
%    \item Essa rappresenta una semplice memoria associativa, ossia una memoria in grado di associare due oggetti. Dato un oggetto, è possible risalire all'altro.
%    \iz {
%      \item Una (banale, vedrete \textbf{molto} di meglio a lezione) strategia implementativa è l'utilizzo di due liste ordinate.
%      \item Quando una nuova associazione viene creata, gli oggetti vengono salvati in due liste diverse nella medesima posizione.
%      \item Quando uno dei due oggetti viene rimosso, viene rimosso anche l'altro
%      \item Quando viene richiesto un oggetto passando l'altro, si cerca in che posizione sia nella lista, quindi si ritorna l'elemento nella stessa posizione dell'altra lista
%    }
%    \item Tenendo conto di quanto detto, si realizzi la classe \cil{oop1415.lab05.exams.generics.AssociativeMemory<K, V>}, che implementa l'interfaccia \cil{oop1415.lab05.exams.generics.IAssociativeMemory<K, V>}.
%  }
%}
%
%\subsection{Ordinamento con Comparable\gen{T}}
%
%\fr {Implementazione di Comparable\gen{T}}{
%  \iz {
%    \item Si implementi \cil{Comparable<IPerson>} in \cil{IPerson}.
%    \item Si implementi in \cil{AbstractPerson} il metodo necessario a supportare \cil{Comparable<IPerson>}. Tale metodo deve ordinare le persone alfabeticamente  per cognome.
%    \item Si ricorda che la classe \cil{String} implementa già \cil{Comparable<String>}: le stringhe hanno ordine naturale uguale all'ordine alfabetico.
%  }
%}
%
%\fr {Test}{
%Si testi il funzionamento di quanto fatto terminando l'implementazione di \texttt{oop1415.lab05.exams.ExamManagerTest}
%}
%
%\subsection{Uso di wildcard bounded}
%\fr {Varianza}{
%  \iz {
%    \item Si rimuavano i commenti la prima parte commentata in \cil{PeopleManagerTest}
%    \item Errori!
%    \item Ragioniamo su cosa stiamo facendo: il metodo vuole prendere una \cil{Collection<IPerson>} per aggiungerla internamente al \cil{PeopleManager}
%    \item Noi stiamo passando una \cil{Collection<Student>}.
%    \item Dato che \cil{Student} implementa \cil{IPerson}, ci aspetteremmo che le \cil{Collection<Student>} siano anche \cil{Collection<IPerson>}, giusto?
%    \item \textbf{NO.}
%    \item Infatti, ad una \cil{Collection<IPerson>} possono essere aggiunti anche dei \cil{Professor}, mentre ad una \cil{Collection<Student>} no.
%  }
%}
%
%\fr {Non ci servono le add: uso di \cil{? extends}}{
%  \iz {
%    \item Nel nostro caso, però, l'uso sarebbe lecito: usiamo la \cil{Collection<Student>} per \textbf{accedere} ai suoi elementi, ma non li aggiungiamo, e l'operazione di aggiungere degli \cil{Student} ad un \cil{IPeopleManager<IPerson>} è lecita.
%    \item In Java, è possibile imporre delle limitazioni per dire al compilatore che utilizzeremo quell'oggetto solo per \textbf{accedere} ai suoi elementi
%    \item Si modifichi il metodo\\ \cil{void addAllFromCollection(Collection<PersonType>)} in: \\ \cil{void addAllFromCollection(Collection<? extends PersonType>)}
%    \item Si osservi che gli errori spariscono
%    \item Si tenti di inserire un nuovo elemento nella \cil{Collection<? extends PersonType>} passata: cosa succede?
%    \item Il compilatore sa che la \cil{Collection} contiene un sottotipo di \cil{PersonType}, ma non sa quale: non vi consente di aggiungere elementi.
%  }
%}
%
%\fr {Contro-varianza}{
%  \iz {
%    \item Si decommenti la seconda parte commentata in \cil{PeopleManagerTest}
%    \item Errori!
%    \item Ragioniamo su cosa stiamo facendo: il metodo vuole prendere una \cil{Collection<IPerson>} per aggiungerci il contenuto del \cil{PeopleManager}
%    \item Noi stiamo passando una \cil{Collection<Object>}.
%    \item Dato che ogni tipo Java è sottotipo di \cil{Object}, ci aspetteremmo che le \cil{Collection<IPerson>} siano anche \cil{Collection<Object>}, giusto?
%    \item \textbf{NO.}
%    \item Infatti, gli elementi di una \cil{Collection<Object>}, non sempre sono delle \cil{IPerson}!
%  }
%}
%
%\fr {Non ci servono le get: uso di \cil{? super}}{
%  \iz {
%    \item Nel nostro caso, però, l'uso sarebbe lecito: aggiungiamo dei PersonType alla \cil{Collection<Object>}, e questo è perfettamente lecito
%    \item In Java, è possibile imporre delle limitazioni per dire al compilatore che utilizzeremo quell'oggetto solo per \textbf{aggiungere} degli elementi, ma non ne controlleremo il contenuto
%    \item Si modifichi il metodo\\ \cil{void addAllToCollection(Collection<PersonType>)} in: \\ \cil{void addAllToCollection(Collection<? super PersonType>)}
%    \item Si osservi che gli errori spariscono
%    \item Si tenti di prendere elemento della \cil{Collection<? super PersonType>} passata: cosa succede?
%    \item Il compilatore sa che la \cil{Collection} contiene un supertipo di \cil{PersonType}, ma non sa quale: vi ritorna un Object.
%  }
%}
%
%\section{Prova d'esame}
%\subsection{Progettazione di una classe generica}
%\fr{Esercizio con classe generica}{
%  \iz {
%    \item Si legga e si risolva l'esercizio in \texttt{UseThree.java}.
%    \item Per problemi, suggerimenti o richieste, si chieda e ci si aiuti anche fra studenti facendo uso del forum del corso.
%  }
%}
%\subsection{Costruzione di un iteratore}
%\fr{Esercizio con iteratore}{
%  \iz {
%    \item Si legga e si risolva l'esercizio nel file testo.txt in \texttt{oop1415.lab05.examsaple2}.
%    \item Per problemi, suggerimenti o richieste, si chieda e ci si aiuti anche fra studenti facendo uso del forum del corso.
%  }
%}

\end{document}



%\subsection{Il classpath}
%\fr{Il classpath}{
%  \iz{
%    \item Il classpath è l'insieme delle directory in cui Java cerca classi e packages.
%    \item Sia \cil{java} che \cil{javac} consentono di specificarlo con l'opzione \cil{-cp}
%  }
%  \bl{In \cil{javac}}{
%    Il classpath indica dove verranno cercate le classi necessarie affinché il programma sia compilato.
%    \iz{
%      \item Uso di librerie esterne
%    }
%  }
%  \bl{In \cil{java}}{
%    Il classpath indica dove verranno cercate le classi necessarie affinché il programma sia eseguito correttamente
%    \iz{
%      \item Uso di librerie esterne
%      \item Software esportato come JAR file
%      \item Risorse da caricare incluse in un JAR file
%    }
%  }
%}
%
%\fr{Compilazione: librerie mancanti}{
%  \iz{
%    \item Ci si posizioni con un terminale nella cartella contenente il progetto
%    \item Si svuoti la cartella \texttt{bin} se presente, la si crei altrimenti
%    \item Si provi a compilare la classe oop1415.lab05.mathlib.TestMathLab (utilizzando \texttt{javac} senza l'opzione \texttt{-cp} come in precedenza, il comando \textbf{dovete} saperlo a questo punto.)
%    \item Cosa succede?
%  }
%  \bl{Leggere e capire gli errori. \textbf{Sempre}.}{
%    \texttt{
%      src/oop1415/lab05/mathlib/TestFastMath.java:6: error: package org.apache.commons.math3.util does not exist
%    }
%  }
%}
%
%\fr{Compilazione: aggiunta di librerie al class path}{
%  \iz{
%    \item Motivo: il programma si appoggia ad una libreria esterna (Apache Commons Math \cite{apache-commons-math})
%    \item È necessario istruire javac dicendogli dove trovare la libreria
%    \item Essa vi viene fornita in \texttt{lib/commons-math3-3.2.jar}
%    \item Si ricompili, aggiungendo l'opzione \texttt{-cp lib/commons-math3-3.2.jar}
%    \item La compilazione va a buon fine!
%  }
%}
%
%\fr{Esecuzione: librerie mancanti}{
%  \iz{
%    \item Lo stesso problema si pone al momento dell'esecuzione
%    \item Dato che avete solo compilato, ma non ``impacchettato'' un JAR che includesse tutto il necessario, è necessario dire alla Java Virtual Machine dove recuperare le classi necessarie per \textbf{eseguire} il programma.
%%     \item Si noti che, in generale, le librerie richieste a tempo d'esecuzione possono essere diverse da quelle necessarie a tempo di compilazione (in genere, di più).
%    \item Ci si posizioni con il terminale dentro la cartella \texttt{bin}.
%    \item Si esegua la classe appena compilata con \texttt{java}, senza usare l'opzione \texttt{-cp} (anche stavolta, dovete essere in grado di scrivere correttamente il comando).
%    \item Cosa succede?
%  }
%  \bl{Leggere e capire gli errori. \textbf{Sempre}.}{
%    \scriptsize\texttt{
%This program will run 1000000 calculations of asin and acos for each library, and print the time. \\
%Java Math took 51.9713719ms to complete. \\
%Exception in thread "main" java.lang.NoClassDefFoundError: \\ org/apache/commons/math3/util/FastMath \\
%        at oop1314.lab05.mathlib.TestFastMath.main(TestFastMath.java:52)
%    }
%  }
%}
%
%\fr{Esecuzione: aggiunta di librerie al class path}{
%  \iz{
%    \item Il programma esegue finché possibile, quindi lancia un errore!
%    \item La Java Virtual Machine, \textbf{a run time}, non riesce a caricare la classe \texttt{org.apache.commons.math3.util.FastMath}
%  }
%  Inclusione della libreria nel classpath
%  \iz{
%    \item Ci si posizioni con un terminale nella cartella del progetto
%    \item Sappiamo che le nostre classi compilate stanno in \texttt{bin}, mentre la libreria necessaria in \texttt{lib/commons-math3-3.2.jar}
%    \item Si provi ad eseguire da questa posizione, utilizzando l'opzione \texttt{-cp} di \texttt{java}:
%    \iz{
%      \item Su sistemi UNIX: \texttt{-cp bin:lib/commons-math3-3.2.jar}
%      \item Su sistemi Windows: \texttt{-cp bin;lib/commons-math3-3.2.jar}
%      \item Ossia: su UNIX i vari componenti del classpath si separano con i due punti, su Windows con il punto e virgola.
%    }
%  }
%}
%
%\fr{Gestione del Classpath da Eclipse}{
%  \iz {
%    \item Eclipse consente di aggiungere le librerie desiderate al classpath
%    \item Si importi il progetto odierno in Eclipse
%    \item Si seguano le istruzioni seguenti per includere nel classpath il file jar necessario alla compilazione
%  }
%} 
%
%\imgfr{Gestione del Classpath da Eclipse}{Properties}{0.4}{img/eclipsejar1}
%\imgfr{Gestione del Classpath da Eclipse}{Java Build Path}{0.8}{img/eclipsejar2}
%\imgfr{Gestione del Classpath da Eclipse}{Libraries}{0.8}{img/eclipsejar3}
%\imgfr{Gestione del Classpath da Eclipse}{Add JARs...}{0.8}{img/eclipsejar4}
%\imgfr{Gestione del Classpath da Eclipse}{I JAR file devono essere dentro uno dei progetti}{0.6}{img/eclipsejar5}
%\imgfr{Gestione del Classpath da Eclipse}{Normalmente, si posizionano in \texttt{lib}}{0.6}{img/eclipsejar6}
%\imgfr{Gestione del Classpath da Eclipse}{}{0.6}{img/eclipsejar7}
%\imgfr{Gestione del Classpath da Eclipse}{}{0.6}{img/eclipsejar8}
%\imgfr{Gestione del Classpath da Eclipse}{}{0.8}{img/eclipsejar9}
%\imgfr{Gestione del Classpath da Eclipse}{Appare ``Referenced Libraries'' nel progetto}{0.99}{img/eclipsejar10}
%\imgfr{Gestione del Classpath da Eclipse}{All'interno, trovate il JAR e tutte le classi in esso contenute!}{0.99}{img/eclipsejar11}


% \subsection{Javadoc personalizzato: doclets}
% 
% \fr{Uso di doclet}{
% Doclets are programs written in the JavaTM programming language that use the doclet API to specify the content and format of the output of the Javadoc tool \cite{doclet}
% 
% 
% }


