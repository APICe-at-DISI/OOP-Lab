\documentclass[presentation]{beamer}
\usepackage{../oop-slides-pianini}
\setbeamertemplate{bibliography item}[text]
\title[OOP02 -- Imperative]{02 \\ Stile, classpath, passaggio di argomenti, programmazione imperativa e costruttori}
\author[Pianini]{Danilo Pianini \\ Angelo Croatti, Simone Grotti, Mirko Viroli}

\begin{document}

\frame[label=coverpage]{\titlepage}

%====================
%Outline
%====================
\begin{frame}<beamer>
	\frametitle{Outline}
	\tableofcontents[]
\end{frame}

\fr{Preparazione ambiente di lavoro}{
	\iz{
		\item Accendere il PC
		\item Loggarsi sul sito del corso
	\iz{
		\item \textcolor{blue}{\url{https://bit.ly/oop2016cesena}}
	}
	\item Scaricare dalla sezione \texttt{lab} del sito il file \texttt{oop-lab02.zip} contenente il materiale dell'esercitazione odierna
	\item Spostare il file scaricato sul Desktop
	\item Decomprimere il file usando 7zip (o un programma analogo) sul Desktop
	}
}

\section{Stile}

\begin{frame}{L'importanza dello stile}
	Cominceremo a correggere anche lo stile con cui scrivete codice
	\begin{itemize}
		\item Il codice che scriverete verrà condiviso con altre persone
		\begin{itemize}
			\item È importante scrivere software immediatamente comprensibile
			\item ``gira'' non è una sufficente metrica di qualità
		\end{itemize}
% 		\item Ogni linguaggio ha le sue prassi
% 		\begin{itemize}
% 			\item Quelle Java di riferimento sono disponibili qui:
% 			\begin{itemize}
% 				\item \url{https://web.archive.org/web/20060228095122/http://developers.sun.com/prodtech/cc/products/archive/whitepapers/java-style.pdf}
% 				\item \url{https://web.archive.org/web/20080513084244/http://java.sun.com/docs/codeconv/CodeConventions.pdf}
% 				\item \url{http://www.oracle.com/technetwork/java/codeconvtoc-136057.html}
% 			\end{itemize}
% 		\end{itemize}
% 		\item Ogni azienda poi è libera di darsi altre regole interne
% 		\begin{itemize}
% 			\item Google: \url{https://google.github.io/styleguide/javaguide.html}
% 			\item Twitter: \url{https://github.com/twitter/commons/blob/master/src/java/com/twitter/common/styleguide.md}
% 			\item Mozilla: \url{https://developer.mozilla.org/en-US/docs/Mozilla/Developer_guide/Coding_Style#Java_practices}
% 		\end{itemize}
		\item È importante adottare uno stile consistente!
		\item Ma bisogna sempre essere consistenti
		\item Nel corso useremo un'estensione di JCC
	\end{itemize}
	\vspace{1cm}
	\begin{quote}
		Always code as if the guy who ends up maintaining your code will be a violent psychopath who knows where you live. Code for readability.
		\flushright{--- John Woods [disputed]}
	\end{quote}
\end{frame}

\begin{frame}{Stili}
	Ogni linguaggio ha le sue prassi
	\begin{itemize}
		\item Quelle Java di riferimento sono disponibili qui:
		\tiny
		\begin{itemize}
			\item \url{http://bit.ly/java-style-guide}
			\item \url{http://bit.ly/java-code-conventions}
			\item \url{http://bit.ly/oracle-java-code-conventions}
		\end{itemize}
	\end{itemize}
	Ogni azienda poi è libera di darsi altre regole interne, ad esempio:
	\footnotesize
	\begin{itemize}
		\item Google: \url{https://google.github.io/styleguide/javaguide.html}
 		\item Twitter: \url{https://github.com/twitter/commons/blob/master/src/java/com/twitter/common/styleguide.md}
 		\item Mozilla: \url{https://developer.mozilla.org/en-US/docs/Mozilla/Developer_guide/Coding_Style\#Java_practices}
	\end{itemize}
	\normalsize
	Notare che sono sempre \textbf{consistenti}!
	\begin{center}
		\textbf{Nel corso useremo un'estensione delle JCC}
	\end{center}
\end{frame}

\begin{frame}{Stili}
	Ogni linguaggio ha le sue prassi
	\begin{itemize}
		\item Quelle Java di riferimento sono disponibili qui:
		\tiny
		\begin{itemize}
			\item \url{http://bit.ly/java-style-guide}
			\item \url{http://bit.ly/java-code-conventions}
			\item \url{http://bit.ly/oracle-java-code-conventions}
		\end{itemize}
	\end{itemize}
	Ogni azienda poi è libera di darsi altre regole interne, ad esempio:
	\footnotesize
	\begin{itemize}
		\item Google: \url{https://google.github.io/styleguide/javaguide.html}
 		\item Twitter: \url{https://github.com/twitter/commons/blob/master/src/java/com/twitter/common/styleguide.md}
 		\item Mozilla: \url{https://developer.mozilla.org/en-US/docs/Mozilla/Developer_guide/Coding_Style\#Java_practices}
	\end{itemize}
	\normalsize
	Notare che sono sempre \textbf{consistenti}!
	\begin{center}
		\textbf{Nel corso useremo un'estensione delle JCC}
	\end{center}
\end{frame}

\begin{frame}[allowframebreaks]{Buone prassi}
	\begin{block}{Usare sempre le parentesi per if, else, for, while, anche se segue una sola istruzione}
		\begin{itemize}
			\item Aumentano la manutenibilità del codice
			\item È facile che nella fretta si modifichi il codice in modo sbagliato
			\item È facile che alcuni tool automatici si sbaglino quando ``uniscono'' pezzi di codice scritti da diverse persone
			\item Apple iOS soffrì di un grave bug a SSL/TLS causato da questa cattiva pratica \url{https://www.imperialviolet.org/2014/02/22/applebug.html}
		\end{itemize}
	\end{block}
	\begin{block}{Le parentesi graffe vanno sempre ``all'egiziana'' (Egyptian brackets)}
		\begin{itemize}
			\item La graffa che apre va in linea con lo statement di apertura
			\item La graffa che chiude va in a capo, nella stessa colonna dello statement di apertura
		\end{itemize}
	\end{block}
	\begin{block}{Naming conventions - \textbf{molto importanti!}}
		\begin{itemize}
			\item I nomi di \texttt{package} usano sempre e solo lettere minuscole
			\item Usare sempre CamelCase, evitare gli underscore (\_)
			\item I nomi di classe cominciano sempre per maiuscola
			\item I nomi di campi, metodi e variabili locali iniziano sempre per minuscola
			\item I campi \texttt{static final} (costanti di classe) sono interamente maiuscoli e possono usare underscore
		\end{itemize}
	\end{block}
	\begin{itemize}
		\item Più avanti avrete ulteriori vincoli
		\item Ma anche degli strumenti automatici che vi aiuteranno
	\end{itemize}
\end{frame}


\section{Compilazione ed esecuzione avanzata di classi Java}

\fr{Nuova opzione per \texttt{javac} \cite{or-javac}}{
	\begin{itemize}
		\item Abbiamo già visto come compilare file sorgenti Java (file \texttt{.java}), generando classi in bytecode, che prendono la forma di file \texttt{.class} nella medesima directory 
		\item Tuttavia è uso comune e \textbf{buona pratica} nella gestione di progetti articolati, \textbf{separare le classi sorgenti dal bytecode}, ad esempio:
	\begin{itemize}
		\item cartella \texttt{src}, per i file sorgenti (\texttt{.java})
		\item cartella \texttt{bin}, contenente le classi compilate (\texttt{.class})
	\end{itemize}
		\item Come si fa?
	\end{itemize}
	\bl{Nuova opzione del comando \texttt{javac}}{
		\iz {
			\item \cil{-d}: consente di specificare la cartella destinazione in cui compilare i file \texttt{.java}
			\item Si tratta di un'opzione che dovete obbligatoriamente saper usare
		}
		\center{\textbf{Sarà oggetto di valutazione in sede di prova pratica!}}
	}
}

\subsubsection{Esempio 1}
\fr{Esempio 1: compilazione nella cartella \texttt{bin}}{
  \en{
      \item Aprire un terminale e posizionarsi nella directory contenente il codice
      \item Spostarsi nella directory \texttt{esempio1}
      \begin{itemize}
         \item contiene la cartella \texttt{src}, in cui è presente \texttt{Calculator.java} (laboratorio scorso)
      \end{itemize}
      \item Creare una cartella \cil{bin}: comando \cil{mkdir bin}
      \item Eseguire la compilazione
      \iz {
	\item su sistemi UNIX (Linux, Mac OSX, BSD):\\ \textbf{\texttt{javac -d bin src/Calculator.java}}
	\item su sistemi Windows:\\ \textbf{\texttt{javac -d bin src\textbackslash{}Calculator.java}}\\
	(Notare l'uso di un differente separatore per il percorso)
      }
      \item Controllate la presenza di \texttt{Calculator.class} all'interno di \texttt{bin}
  }
}

\fr{Esempio 1: compilare più file contemporaneamente da diverse locazioni}{
  \begin{block}{Compilazione in directory arbitrarie}
    \texttt{javac -d <CARTELLA DESTINAZIONE> <FILE JAVA>}
    \begin{itemize}
      \item \tiny{\textbf{OVVIAMENTE} vanno sostituite le variabili fra parentesi angolari con le directory che andranno usate.}
    \end{itemize}
  \end{block}
  \begin{itemize}
    \item Si cancelli il contenuto di \texttt{bin}
    \item Si posizioni il terminale nella root (\texttt{/} per Unix, \texttt{C:\textbackslash} per Windows) e si ricompili \texttt{Calculator.java} dentro la medesima cartella \texttt{bin}
  \end{itemize}

  \begin{block}{Compilazione di più file in una singola passata}
    \texttt{javac -d <CARTELLA DESTINAZIONE> <ELENCO DI FILE JAVA>}
    \begin{itemize}
      \item \tiny{\textbf{OVVIAMENTE} vanno sostituite le variabili fra parentesi angolari con le directory che andranno usate.}
    \end{itemize}
  \end{block}
  È possibile anche utilizzare la wildcard (\texttt{*}) invece di elencare tutti i file!
  \begin{itemize}
    \item Su Unix si possono usare wildcard in più punti del path, ad esempio \texttt{progetti/*/src/*.java} elenca tutti i file con estensione java dentro ciascuna cartella \texttt{src} di ciascuna cartella dentro \texttt{progetti}
  \end{itemize}
}

\fr{Esempio 1: Esecuzione di \texttt{Calculator} }{
  \iz{
      \item Possiamo usare la ``strategia'' già vista nello scorso laboratorio:
      \begin{enumerate}
         \item spostarsi in \texttt{bin}
         \item eseguire: \texttt{java Calculator}
      \end{enumerate}
      \item \textbf{Esiste un modo molto più flessibile!}\\(l'unico veramente efficace quando lavorerete con i package)
  }
  \bl{Nuova opzione del comando \texttt{java} \cite{or-java}}{
    \iz {
      \item \cil{-cp} (oppure \cil{-classpath}): consente di specificare il \textbf{classpath}, ovvero il percorso in cui cercare i file \texttt{.class} da eseguire
      \item Va da sé\ldots
      \center{\textbf{\ldots sarà oggetto di valutazione in sede di prova pratica!}} 
    }
  }
  \bl{Esecuzione tramite classpath esplicito}{
      \en{
         \item Posizionarsi in \texttt{esempio1}
         \item eseguire: \texttt{java -cp bin Calculator}
      }
  }
}
\fr{Il classpath in dettaglio}{
	\iz{
		\item Il classpath è l'insieme delle directory in cui Java cerca le classi
		\item Sia \cil{java} che \cil{javac} consentono di specificarlo con l'opzione \cil{-cp}
	}
	\bl{In \cil{java}}{
		Il classpath indica dove verranno cercate le classi necessarie affinché il programma sia eseguito correttamente
	}
	\bl{In \cil{javac}}{
		Il classpath indica dove verranno cercate le classi necessarie affinché il programma sia compilato
	}
}
\begin{frame}{Più cartelle nel classpath}
	Il classpath non è composto da una sola cartella, ma può contenere più cartelle, oltre a file compressi contenenti bytecode e risorse, come zip o jar (li vedremo in futuro).
	\begin{block}{}
		Si possono specificare più cartelle utilizzando come separatore il simbolo ``\texttt{:}'' (per sistemi Unix) oppure ``\texttt{;}'' (per sistemi Windows)
		\begin{itemize}
			\item \texttt{javac -d bin -cp lib1:lib2:lib3 src/*.java}
			\begin{itemize}
				\item Compila tutti i file con estensione java che si trovano nella cartella \texttt{src}, mettendo i compilati dentro \texttt{bin}. In compilazione, potrà linkare tutte le classi che si trovano nelle cartelle \texttt{lib1}, \texttt{lib2} e \texttt{lib3}: nel caso in cui alcuni sorgenti in src stiano usando delle classi definite dentro queste cartelle, la compilazione avrà successo.
				\item \scriptsize{Equivalente Windows: \texttt{javac -d bin -cp lib1;lib2;lib3 src/*.java}}
		\end{itemize}
			\item \texttt{java -cp bin:lib1:lib2:lib3 MyClass}
			\begin{itemize}
				\item Esegue il main della classe \texttt{MyClass}. Cercherà questa classe e tutte quelle collegate all'interno delle cartelle \texttt{bin}, \texttt{lib1}, \texttt{lib2} e \texttt{lib3}.
				\item \scriptsize{Equivalente Windows: \texttt{java -cp bin;lib1;lib2;lib3 MyClass}}
			\end{itemize}
		\end{itemize}
	\end{block}
\end{frame}

\subsubsection{Esempio 2}
\begin{frame}{Esempio 2: Compilazione ed esecuzione di classi strutturate in package}
	\begin{itemize}
		\item Spostarsi nella directory \texttt{esempio2}, contenente il secondo esempio
		\item All'interno è presente la cartella \texttt{src}, come nel precedente esempio 
		\item In questo caso src non contiente direttamente una classe java ma una \textbf{gerarchia di cartelle}: \texttt{oop/lab02/math}\\
		\begin{beamerboxesrounded}[shadow=true]{}
			È buona norma organizzare i sorgenti in modo da rappresentare su filesystem la struttura dei package. Si noti però che (dato che il compilatore lavora su \alert{file}) questa scelta \textbf{non è obbligatoria}!
		\end{beamerboxesrounded}
		\item Nella cartella ``foglia'' \texttt{math} è presente il sorgente \texttt{ComplexNum.java} (scorso laboratorio)
	\end{itemize}
\end{frame}

\fr{Esempio 2: la classe \texttt{ComplexNum}}{
   \sizedcode{\small}{code/esempio2/src/oop/lab02/math/ComplexNumShort.java}
   \bx{\center{\textbf{\texttt{ComplexNum} è ora definita all'interno di un package}}}
}

\begin{frame}{Esempio 2: compilazione/utilizzo di javac}
	\begin{enumerate}
		\item Spostarsi (se non lo si è già fatto) in \texttt{esempio2}
		\item Eseguire la compilazione:
		\begin{itemize}
			\item su sistemi UNIX (Linux, Mac OSX, BSD):\\ \textbf{\texttt{javac -d bin src/oop/lab02/math/ComplexNum.java}}
			\item su sistemi Windows:\\ \textbf{\texttt{javac -d bin src\textbackslash{}oop\textbackslash{}lab02\textbackslash{}math\textbackslash{}ComplexNum.java}}
		\end{itemize}
		\item Controllare ora il contenuto di \texttt{bin}: cosa vi aspettate di trovare?
		\begin{itemize}
			\item La struttura dei package è stata riprodotta dal compilatore
			\item Dato che l'interprete non lavora con file ma con \alert{classi}, il loro layout sul file system \textbf{non può essere modificato!}
% 			\item Provate a spostare il file generato dal compilatore in una cartella differente: non riuscirete più ad eseguire la classe ivi scritta
		\end{itemize}
	\end{enumerate}
\end{frame}

\fr{Esempio 2: esecuzione/uso di java}{
  \iz{
      \item L'esecuzione funziona come già visto nel primo esempio, tramite \textbf{opzione \texttt{-cp}}

      \item Eseguire la classe
      \iz {
	\item \texttt{java -cp bin ComplexNum}
      }
   }
   \bx{\center{La classe viene eseguita?}}
   
}

\fr{Esempio 2: esecuzione/uso di java}{
  \iz{
      \item L'esecuzione funziona come già visto nel primo esempio, usando \textbf{l'opzione \texttt{-cp}}

      \item Eseguire quindi la compilazione:
      \iz {
	\item \texttt{java -cp bin ComplexNum}
      }
   }
   \bx{\center{\textbf{No! Bisogna usare l'intero nome della classe, che in Java include anche il nome del package}}}  
}

\begin{frame}{Esempio 2: esecuzione/uso di java}
	\begin{itemize}
		\item Eseguendo: \texttt{java -cp bin ComplexNum}
		\item La JVM si aspetterebbe di la classe \texttt{ComplexNum} nel package di default dentro il classpath, composto nel nostro caso dalla sola directory \texttt{bin}
		\item Ossia, per il modo in cui sono rappresentate le classi nel file system, cercherebbe un file \texttt{ComplexNum.class} dentro \texttt{bin}
	\end{itemize}
   \bx{\center{Qual è dunque il metodo corretto per eseguire con successo \texttt{ComplexNum}?}} 

\end{frame}

\fr{Esempio 2: esecuzione/uso di java}{
	\iz{
		\item Proviamo con:
		\iz {
			\item \texttt{java -cp bin oop.lab02.math.ComplexNum}
		}
		\item Questa volta abbiamo avuto successo!
		\item la JVM, all'interno di \texttt{bin}, si aspetta la struttura \texttt{oop/lab02/math} contenente \texttt{ComplexNum.class}
		\item Dato che l'interprete non lavora con file ma con \alert{classi}, il loro layout sul file system \textbf{non può essere modificato!}
		\item Provate a spostare il file generato dal compilatore in una cartella differente: non riuscirete più ad eseguire la classe.
	} 
}
\subsubsection{Esempio 3}
\fr{Esempio 3: compilazione ed esecuzione ``avanzati''}{
  \iz{
      \item Spostarsi nella directory \texttt{esempio3}
      \item All'interno è presente la solita cartella \texttt{src} 
      \item Il contenuto corrisponde alla \textbf{gerarchia di cartelle}: \texttt{oop/lab02/math}
      \item Nello specifico, dentro \texttt{math}, è presente il sorgente \texttt{UseComplex.java} (scorso laboratorio)
   }
}

\fr{Esempio 3: la classe \texttt{UseComplex}}{

   \sizedcode{\tiny}{code/esempio3/src/oop/lab02/math/UseComplex.java}
}

\fr{Esempio 3: compilazione}{
  \en{
      \item Spostarsi (se non li si è già fatto) in \texttt{esempio3}
      \item Eseguire la compilazione:
      \iz {
	\item Su sistemi UNIX (Linux, Mac OSX, BSD):\\ \textbf{\texttt{javac -d bin src/oop/lab02/math/UseComplex.java}}
	\item Su sistemi Windows:\\ \textbf{\texttt{javac -d bin src\textbackslash{}oop\textbackslash{}lab02\textbackslash{}math\textbackslash{}UseComplex.java}}
      }
      \item Otteniamo un errore!
   }
   \bl{Comprensione degli errori}{\sizedcode{\tiny}{code/esempio3/src/oop/lab02/math/error.txt}}
   
}

\fr{Esempio 3: compilazione}{
	\bl{Comprensione degli errori}{\sizedcode{\tiny}{code/esempio3/src/oop/lab02/math/error.txt}}
	\iz{
		\item Il compilatore ha bisogno di conoscere la classe ComplexNum per poterla linkare e per poter compilare una classe che la riferisce
		\item Il compilatore cerca nel classpath il bytecode della classe \texttt{ComplexNum}
		\item Come risolviamo?
	}
	\bx{\center{\textbf{Utilizzo dell'opzione \texttt{-cp} sul comando \texttt{javac}!}}} 
}

\fr{Esempio 3: compilazione}{
     \iz{
      \item Nell'esempio precedente avevamo già compilato \texttt{ComplexNum}:
       \iz{
       	\item \texttt{ComplexNum.class} si trova nella cartella \texttt{bin} all'interno di \texttt{esempio2}
       }
       \item Eseguiamo nuovamente la compilazione
      \iz {
	\item Su sistemi UNIX (Linux, Mac OSX, BSD):\\
	\bx{\scriptsize{\texttt{javac -d bin -cp ../esempio2/bin src/oop/lab02/math/UseComplex.java}}}
	\item Su sistemi Windows:\\
	\bx{\scriptsize{\texttt{javac -d bin -cp ..\textbackslash{}esempio2\textbackslash{}bin src\textbackslash{}oop\textbackslash{}lab02\textbackslash{}math\textbackslash{}UseComplex.java}}}\normalsize
      }
      \bx{Contenuto di \texttt{bin}: file \texttt{UseComplex.class} all'interno di \texttt{oop/lab02/math}}
     }
   
}
\fr{Esempio 3: esecuzione di \texttt{UseComplex}}{
     \iz{
      \item Per eseguire correttamente \texttt{UseComplex} dobbiamo dire alla JVM, tramite \textbf{-cp}, dove trovare:
       \iz{
       	\item \texttt{ComplexNum.class}
        \item \texttt{UseComplex.class}
       }
       \item Si trovano in \texttt{due percorsi diversi}!
       \item Dobbiamo specificare come argomento di \texttt{-cp} due percorsi, usando il \textbf{separatore}:
       \iz {
	\item \textbf{\texttt{:}} su sistemi UNIX (Linux, Mac OSX, BSD) 
	\item \textbf{\texttt{;}} su sistemi Windows
       }
     }
}
\fr{Esempio 3: secuzione di \texttt{UseComplex}}{
  \iz{
      \item Eseguiamo il comando:
      \iz {
	\item sistemi UNIX (Linux, Mac OSX, BSD):\\ \bx{\texttt{java -cp bin:../esempio2/bin oop.lab02.math.UseComplex}}
	\item sistemi Windows:\\ \bx{\texttt{java -cp bin;..\textbackslash{}esempio2\textbackslash{}bin oop.lab02.math.UseComplex}}
      }
   } 
   Si ricorda che \alert{\texttt{..}} significa ``livello directory superiore a quello corrente'', mentre \alert{\texttt{.}} significa ``livello directory corrente''
}
\subsubsection{Esempio 4}
\fr{Esempio 4: Compilazione da più cartelle sorgenti}{
  \iz{
      \item Spostarsi nella directory \texttt{esempio4}
      \item All'interno è presente la solita cartella \texttt{src}, contenente \texttt{UseCalculator.java} (scorso laboratorio)
   }
}

\fr{Esempio 4: la classe \texttt{UseCalculator}}{

   \sizedcode{\small}{code/esempio4/src/UseCalculator.java}
   \bx{Esegue il test di \texttt{Calculator}}
}

\fr{Esempio 4: compilazione}{
  \en{
      \item Come nel caso precedente, \texttt{UseCalculator.java} fa riferimento ad un'altra classe (\texttt{Calculator.java})
      \item Stavolta le compileremo entrambe, \textbf{contemporaneamente}
      \iz {
	\item Su sistemi UNIX (Linux, Mac OSX, BSD):\\ \bx{\tiny\textbf{\texttt{javac -d bin src/UseCalculator.java ../esempio1/src/Calculator.java}}}
	\item Su sistemi Windows:\\ \bx{\tiny\textbf{\texttt{javac -d bin src\textbackslash{}UseCalculator.java ..\textbackslash{}esempio1\textbackslash{}src\textbackslash{}Calculator.java}}}
      }
   }
      \bx{Contenuto di \texttt{bin}: file \texttt{UseCalculator.class} e \texttt{Calculator.class}}
}

\fr{Esempio 4: esecuzione}{

      \bx{\center{\Large Questa volta fate voi!}}\normalsize
}

\begin{frame}{Consiglio finale}
  Visto che all'esame il loro utilizzo è richiesto, è necessario imparare \textbf{a memoria} le opzioni di \texttt{java} e \texttt{javac}?
  \begin{center}
    \textbf{NO}
  \end{center}
  Entrambi i comandi (e praticamente tutti i comandi Unix) hanno con loro un'opzione che consente di stampare a video un help. Provate
  \begin{block}{}
    \begin{itemize}
      \item java -help
      \item javac -help
    \end{itemize}
  \end{block}
  \vspace{10pt}
  
  Gli help stampano una sbrodolata di testo e a me serve una riga, davvero devo imparare a leggere e capire un help?
  \begin{center}
    \textbf{SÌ}
  \end{center}
  È molto facile dimenticarsi la sintassi delle opzioni di comandi che non si usano spesso. È molto più facile imparare a destreggiarsi in un help che andare a tentativi o ricordare cose a memoria.
\end{frame}

\section{Passaggio degli argomenti}

\begin{frame}{Passaggio degli argomenti}
	\begin{itemize}
		\item La maggior parte dei comandi supporta degli argomenti
		\begin{itemize}
			\item Ad esempio, quando eseguite \texttt{javac -d bin MyClass.java} gli argomenti sono:
			\begin{enumerate}
			 \item \texttt{-d}
			 \item \texttt{bin}
			 \item \texttt{MyClass.java}
			\end{enumerate}
		\end{itemize}
		\item In C, questi vengono passati al \texttt{main} come coppia di \texttt{char **} e \texttt{int}, rappresentanti rispettivamente un riferimento all'area di memoria dove sono salvati i parametri ed il numero dei suddetti.
		\item Anche in Java ovviamente è possibile passare degli argomenti ad un programma 
		\item La gestione è un po' più semplice che in C, grazie al fatto che gli array si portano dietro l'informazione circa la loro dimensione
		\item E grazie al fatto che la signature del main è una sola.
		\begin{itemize}
			\item \texttt{public static void main(String [])} è l'unica signature valida
			\item In C sia \texttt{int main(void)} che \texttt{int main(char **, int)} sono accettabili
		\end{itemize}
	\end{itemize}
\end{frame}

\begin{frame}{Argomenti in Java}
	\begin{itemize}
		\item Gli argomenti con cui il programma Java viene invocato vengono inseriti dentro lo \texttt{String[]} che il \texttt{main} prende in ingresso
		\item Questo è un normalissimo array di \texttt{String}, uguale a qualunque altro array di \texttt{String}, e senza alcuna specialità
	\end{itemize}
	Provate a svolgere l'esercizio proposto come esempio 5 nel codice!
\end{frame}


\section{Esercizi in laboratorio}


\fr{Modalità di Lavoro}{
	\bl{}{
               Gli esercizi si trovano nella cartella \texttt{esercizi}
		\en{
			\item Leggere la consegna
			\item Risolvere l'esercizio in autonomia
			\item Cercare di risolvere autonomamente eventuali piccoli problemi che possono verificarsi durante lo svolgimento degli esercizi
			\item \alert{Utilizzare le funzioni di test presenti nei sorgenti per il testing dell'esercizio}
			\item Contattare i docenti nel caso vi troviate a lungo bloccati nella risoluzione di uno specifico esercizio
			\item A esercizio ultimato contattare i docenti per un rapido controllo della soluzione realizzata
			\item Proseguire con l'esercizio seguente
		}
	}
}

\subsection{Uso dei costruttori}

\fr{Esercizi sui costruttori 1/2}{
        \bx{Fate riferimento alla parte sui costruttori nelle slide viste durante l'ultima lezione: \texttt{04 - Oggetti e classi, pp 10-18}}
	\bl{Implementare i costruttori per alcune classi della prima esercitazione}{
		\en {
			\item In \cil{oop.lab02.constructors} sono presenti le classi \cil{Train e Student} relative all'esercitazione della settimana scorsa
			\item Rimuovere dalle classi il metodo \cil{build} e il metodo \cil{main}
			\item Implementare due diversi costruttori per la classe \cil{Train}
				\iz {
					\item Un costruttore con zero parametri
						\iz {
							\item I campi della classe dovranno essere inizializzati con dei valori di default (sensati) scelti dallo studente
						}
					\item Un costruttore che specifica l'insieme di parametri necessario ad inizializzare i campi della classe
				}
			\item Implementare in autonomia il costruttore della classe Student
			\item Seguire le istruzioni riportate nei commenti di \cil{UseConstructors} per effettuare il test dei nuovi costruttori
		}
	}
}

\fr{Esercizi sui costruttori 2/2}{
        \bx{
          Fare riferimento a quanto discusso a lezione, sulla possibilità di \emph{richiamare costruttori esistenti nel corpo di un altro costruttore tramite istruzione}
          \begin{center}
            \cil{this(...)}
          \end{center}
        }
	\bl{Modificare/aggiungere i seguenti costruttori alla classe \cil{Train}}{
		\iz {
			\item Modificare il costruttore a zero parametri \cil{Train()}, in modo da riusare quello che accetta parametri di inizializzazione
                        \iz {
				\item Si riutilizzino in modo appropriato i valori di default definiti in precedenza
			}
			\item Un costruttore \cil{Train(int nFCSeats, int nSCSeats)} che richiami, \emph{in modo sensato}, uno dei costruttori definito in precedenza, in modo da inizializzare in \emph{modo consistente} lo stato degli oggetti di classe \cil{Train}
			\item Seguire le istruzioni riportate nei commenti di \cil{UseConstructorsEvolved} per effettuare il test dei nuovi costruttori
		}
	}
}


\subsection{Costruzione di semplici algoritmi su array}
\fr{Semplici Algoritmi su Array (\cil{oop.lab02.arrays}) 1/2}{
		\bl{Analizzare il metodo \cil{countOccurr} della classe \cil{WorkWithArrays}}{
			\iz{
				\item Capire il comportamente realizzato aiutandosi analizzando il metodo di testing \cil{boolean testCountOccurr()} presente nel main
				\item Compilare e eseguire la classe
				\item Aggiungere un nuovo test al metodo \cil{testCountOccurr}
				\item Compilare e eseguire la classe nuovamente
			}
		}
}

\fr{Semplici algoritmi su array (\cil{oop.lab02.arrays}) 2/2}{
	\bl{Completare i seguenti metodi statici nella classe \cil{WorkWithArrays}}{
		\iz{
			\item \cil{public static int[] evenElems(int[] array)}
			\iz{
				\item Restituisce un array con i soli elementi in posizione (indice) \textit{pari} dell'array fornito in input  p.e [1,2,3,4] $\rightarrow$ [1,3]
			}
			\item \cil{public static int[] oddElems(int[] array)}                                                                                                                                       
			\iz{
				\item Restituisce un array con i soli elementi in posizione (indice) \textit{dispari} dell'array fornito in input p.e [1,2,3,4] $\rightarrow$ [2,4]
				
			}
			\item \cil{public static int getMostRecurringElem(int[] array)}
			\iz{
				\item Restituisce l'elemento che viene ripetuto più volte all'interno di \cil{array}
				\item p.e. [1,2,1,3,4] $\rightarrow$ 1; [7,1,5,7,7,9] $\rightarrow$ 7
			}
		}
	}
	
	\bl{}{Per il testing appoggiarsi gli appositi metodi già presenti nella classe \cil{WorkWithArrays}
	}
}

\section{Esercizi avanzati}

\subsection{Esercizi avanzati sui costruttori}
\fr{Esercizi  avanzati sui costruttori}{
   \bl{Completare la classe \cil{Smartphone} in \cil{oop.lab02.constructors}}{
      Classe atta a descrivere un generico smartphone. Campi:
      \iz{
         \item \cil{int nCPU, ram, sdSize}
         \item \cil{String brand, model} 
         \item \cil{boolean hasGPS, hasNFC, has3G} 
      }
   }
   \bl{Implementare i seguenti costruttori per la classe}{
      \iz{
         \item \cil{Smartphone(String brand, String model)}
         \item \cil{Smartphone(String brand, String model, int sdSize)}
         \item \cil{Smartphone(<param. precedenti>, boolean hasGPS, has3g)}
         \item \cil{Smartphone(String brand, String model, boolean hasNFC)}
         \item Un costruttore che richiede come parametri tutti i campi
      }
   }
   \center{Ricordarsi che si possono richiamare costruttori esistenti nel corpo di un altro costruttore tramite istruzione \cil{this(..)} (vista a lezione)}
}

\fr{Testing degli esercizi avanzati sui costruttori}{
   \bl{Testing}{
      Seguire le linee guida riportate nei commenti del \cil{main} della classe \cil{Smartphone}
   }
}

\subsection{Esercizi avanzati sugli array}
\fr{Classe \cil{oop.lab02.arrays.SeqRecognizer } 1/6}{
		\bl{Classe \cil{SeqRecognizer}: riconoscitore di varie sequenze di caratteri}{Analizzare un primo esempio concreto (già implementato) per la sequenza così descritta %grammatica {1}{2|3}
			\iz{
				\item 1: 0-N occorrenze, poi
				\item 2 o 3: 0-N occorrenze
			}
		}
		\bl{Strategia risolutiva in linguaggio naturale}{
			\en{
					\item Finché trovo una sequenza di uno, vado avanti nell’array
					\item Finché trovo dei due o dei tre, vado avanti nell’array
					\item	Verifico di essere arrivato in fondo
			}
		}
		\bx{
			Analizzare l'implementazione del riconoscitore definita nel metodo \cil{boolean checkSeq1(int[] array)}
		}
}

\fr{Classe \cil{oop.lab02.arrays.SeqRecognizer } 2/6}{
		\bl{}{Completare i seguenti metodi statici nella classe \cil{SeqRecognizer}}
		\bl{\cil{public static boolean checkSeq2(int[] array)}}{
			\iz{
				\item Riconoscitore della sequenza % grammatica 1{2}3
					\iz{
						\item	1: una sola occorrenza, poi
						\item 2: 0-N occorrenze, poi
						\item 3: una sola occorrenza
					}
			}
		}
		\bl{\cil{public static boolean checkSeq3(int[] array)}}{
			\iz{
				\item Riconoscitore della sequenza % grammatica 1{2}3{4}[5]
					\iz{
						\item	1: una sola occorrenza, poi
						\item 2: 0-N occorrenze, poi
						\item 3: una sola occorrenza, poi
						\item 4: 0-N occorrenze, poi
						\item 5: 0-1 occorrenze
					}
			}
		}
}


\fr{Esercizi avanzati 3/6}{
	\bl{Implementare i seguenti metodi statici nella classe \cil{WorkWithArrays}}{
		\iz{
			\item \cil{public static int[] sortArray(int[] array,boolean desc)}
			\iz{
				\item Ordina l'array fornito in input (ordinamento crescente/descrescente a seconda del valore del parametro \cil{desc})
				\item Potete far riferimento all'algoritmo bubble sort 
			}
			\item \cil{public static double computeVariance(double[] array)}
			\iz{
				\item Calcola la varianza ($\sigma^2 = \frac{\displaystyle\sum_{i=1}^{n}(x_i - \mu)^2} {n}$) dell'array passato in input 
			}
		}
	}
}

\fr{Esercizi avanzati 4/6}{
	\bl{Implementare i seguenti metodi statici nella classe WorkWithArrays}{
		\iz{
			\item \cil{public static int[] revertUpTo(int[] array,int stopEl)}
				\iz{
					\item Inverte gli elementi di \cil{array} dall'inizio fino all'elemento \cil{stopElem} \textbf{incluso}. Gli elementi successivi a \cil{stopEl} devono rimanere invariati
					\item p.e. revertUpTo([1,2,3,4,5,6,7],5) $\rightarrow$ [5,4,3,2,1,6,7]
				}		
			\item \cil{public static int[] dupElems(int[] array, int nTimes)}
				\iz{
					\item Restituisce un array in cui ciascun cun elemento dell'array in input è ripetuto \cil{nTimes} volte in sequenza
					\item p.e. dupElems([1,2,3],2) $\rightarrow$ [1,1,2,2,3,3]
					\item p.e. dupElems([1,2],5) $\rightarrow$ [1,1,1,1,1,2,2,2,2,2]; ...
				}
		}
	}				
	\bl{Implementare il seguente nuovo metodo nella classe \cil{SeqRecognizer}}{
		\cil{public static boolean checkSeq4(int[] array)}
			\iz{
				\item Riconoscitore della sequenza % grammatica (2|3){4}5
					\iz{
						\item 2 o 3: 0-1 occorrenze
						\item 4: 0-N occorrenze
						\item 5 una occorrenza
					}
			}
	}
}

\fr{Esercizi avanzati 5/6}{
		\bl{\cil{oop.lab02.arrays.PlayWithSmartphoneArrays}}{
			La classe dovrà consentire di realizzare semplici operazioni su array di \cil{Smartphone} (classe costruita in precedenza)
		}
		\bl{Metodi da completare}{
			Deve essere restituito il numero di smartphones che rispettano le preferenze fornite in input
			\iz{
				\item \cil{static int search(Smartphone[] array, String brand)}
				\item \cil{static int search(Smartphone[] array, boolean hasNFC, boolean hasGPS)}
				\item \cil{static int search(Smartphone[] array, int nCPUs, int sdSize, boolean has3G)}
			}
		}
		\bx{
			Per il testing seguire le linee guida definite nel metodo \cil{main}
		}
}
\fr{Esercizi avanzati 6/6}{
	\bl{Completare la classe \cil{oop.lab02.arrays.MyCircularArray}}{
		Realizza un array circolare
		\iz{
			\item Un campo \cil{int[] array}
			\item \cil{public MyCircularArray(int size)} (costruttore)
			\item \cil{public MyCircularArray()} (costruttore)
			\item \cil{public void addElem(int elem)}
				\iz{
					\item Aggiunge l'elemento \cil{elem} all'interno di \cil{array} in modo che, una volta raggiunta la dimensione massima dell'array, i nuovi valori vadano a sovrascrivere quelli precedentemente memorizzati
				}

			\item \cil{public void reset()}
				\iz{
					\item Re-inizializza a zero tutti gli elementi
				}
		}
	}
	\bl{}{
		Seguire le linee guida presenti nel \cil{main} per il test
	}
}


\begin{frame}[allowframebreaks]
 \frametitle{Bibliography}
	\bibliographystyle{plain}
	\small
 \bibliography{biblio}
\end{frame}


\end{document}
