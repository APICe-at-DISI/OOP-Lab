%\documentclass[12pt,handout]{beamer}
\documentclass[presentation]{beamer}
\usepackage{oop-slides}
\setbeamertemplate{bibliography item}[text]
\title[OOP01 -- Intro]{01 \\ Introduzione al Laboratorio e Tool di Base}

\begin{document}

\frame[label=coverpage]{\titlepage}

%====================
%Outline
%====================
\fr{Outline}{
  \bl{Goal della lezione}{
    \iz{ 
			\item Overview su Java (linguaggio + piattaforma)
			\item Compilazione ed esecuzione di programmi Java a riga di comando
      \item Ingegnerizzazione e test di semplici classi
    }
  }
}
\begin{frame}<beamer>
 	\frametitle{Outline}
 	\tableofcontents[]
\end{frame}
  
\section{Java Overview}

%====================
%Java nel mondo del lavoro
%====================
\fr{Java nel mondo ``reale'' 1/2}{
	\fg{height=0.70\textheight}{img/javaTrend.png}
}

\fr{Java nel mondo ``reale'' 2/2}{
	\fg{height=0.70\textheight}{img/javaTrend_highlight.png}
}
		
%====================
%Architettura a Runtime
%====================
\fr{Architettura a Runtime}{
	\fg{height=0.80\textheight}{img/arch_runtime.png}
}




%====================
%JDK
%====================

\fr{Java Development Kit (JDK)}{
	
	\bl{Esistono diverse distribuzioni di Java}{
		\begin{description}
			\item[JRE] -- Java Runtime Environment: esecuzione di programmi
			\item[JDK] -- JRE + tool per lo sviluppo di programmi (e.g. compilatore)
			\item[J2EE] -- Java Enterprise Edition: JDK + set esteso di librerie
		\end{description}
	}
	
	\bl{Noi faremo riferimento al JDK \cite{jdk-download-page}}{Include il necessario per eseguire applicazioni	Java (JRE), i tool (e librerie) per sviluppare applicazioni e la relativa  documentazione}
	
	\bl{I Principali Tool}{
		\begin{description}
			\item[javac] -- Compilatore Java
			\item[java] -- Java virtual machine, per eseguire applicazioni Java
			\item[javadoc] -- Per creare automaticamente documentazione in HTML
			\item[jar] -- Creazione di archivi compressi (anche eseguibili) contenenti bytecode e risorse
		\end{description}
	}
}

\section{Richiamo: il file system}
\begin{frame}{Elementi base del file system}
  \begin{itemize}
    \item I sistemi operativi odierni consentono di memorizzare permanentemente le informazioni su supporti di memorizzazione di massa (dischi magnetici, dispositivi a stato solido), unità ottiche (CD, DVD, Blu-Ray), memory stick, ecc...
    \item Le informazioni su questi supporti sono organizzate in file e cartelle:
      \begin{itemize}
        \item i file contengono le informazioni
        \item le cartelle sono contenitori, all'interno contengono i file ed altre cartelle
      \end{itemize}
    \item La cartella più esterna, che contiene tutte le altre, è detta root. Essa rappresenta il livello gerarchico più alto del file system
      \begin{itemize}
        \item In Windows, ciascun file system ha come root una lettera di unità (e.g. \texttt{C:}, \texttt{D:})
        \item In *nix (Linux, MacOS, BSD, Solaris...), vi è una unica radice, ossia \texttt{/}
      \end{itemize}
    \item La stringa che descrive un intero percorso dalla root fino ad un elemento del file system prende il nome di \emph{percorso} (e.g. \texttt{C:\textbackslash{}Windows\textbackslash{}win32.dll}, \texttt{/home/user/frameworkFS.jar})
  \end{itemize}
\end{frame}

\begin{frame}[fragile]{Manipolare il file system}
  L'utente può osservare e manipolare il file system:
  \begin{itemize}
    \item sapere quali files e cartelle contiene una cartella
    \item creare nuovi files e cartelle
    \item spostare file e cartelle dentro altre cartelle
    \item rinominare files e cartelle
    \item eliminare files e cartelle
  \end{itemize}
  Il software che consente di osservare e manipolare il file system prende il nome di \alert{file manager}.
  \begin{itemize}
    \item Su Windows, esso è ``Esplora risorse'' (\texttt{explorer.exe})
    \item Su MacOS, il principale è ``Finder''
    \item Su Linux (e Android) ne esistono diversi (Nautilus, Dolphin, Thunar, Astro...)
  \end{itemize}
\end{frame}




\section{Richiamo: Interprete Comandi}
\fr{Interprete Comandi}{
  \bl{}{
    Programma che permette di interagire con il S.O. mediante comandi impartiti in modalità testuale (non grafica), via linea di comando
  }
% 	\bl{Scripting}{
% 		L'interprete comandi può avere un linguaggio associato con cui è possibile scrivere script
% %		\iz{
% %			\item Utili (principalmente) per automatizzare task di diversa natura
% %		}
% 	}
% 	\bl{Vari tipi di comandi}{
% 		\iz{
% 			\item Navigazione file system
% 			\item Interazione con il file system
% 			\item Esecuzione di programmi da riga di comando
% 			\item ...
% 		}
% 	}
  \begin{itemize}
    \item Nell'antichità (in termini informatici) le interfacce grafiche erano sostanzialmente inesistenti, e l'interazione con i calcolatori avveniva di norma tramite interfaccia testuale
    \item Tutt'oggi, le interfacce testuali sono utilizzate:
    \begin{itemize}
      \item per automatizzare le operazioni
      \item per velocizzare le operazioni (scrivere un comando è spesso molto più veloce di andare a fare click col mouse in giro per lo schermo)
      \item per fare operazioni complesse con pochi semplici comandi
      \item non tutti i software sono dotati di interfaccia grafica
      \item alcune opzioni di configurazione del sistema operativo restano accessibili solo via linea di comando
      \begin{itemize}
        \item (anche su Windows: ad esempio i comandi per associare le estensioni ad un eseguibile)
      \end{itemize}
    \end{itemize}
  \end{itemize}
  \bl{}{
    Lo vedrete in maniera esaustiva nel corso di Sistemi Operativi...
  }
}

\fr{Sistemi *nix (Linux, MacOS X, FreeBSD, Minix...)}{
        \bl{Nei sistemi UNIX esistono vari tipi di interpreti, chiamati shell}{
                Alcuni esempi
                \iz{
                        \item Bourne shell (sh)
                                \iz{ \item Prima shell sviluppata per UNIX (1977)}
                        \item C-Shell (csh)
                                \iz{ \item Sviluppata da Bill Joy per BSD}
                        \item Bourne Again Shell (bash)
                                \iz{ \item Parte del progetto GNU, è un super set di Bourne shell}
                        \item ...
                }
        }
        \bl{Per una panoramica completa delle differenze}{
                \textcolor{blue}{\url{http://www.faqs.org/faqs/unix-faq/shell/shell-differences/}}
        }
}

\fr{Sistemi Windows}{
  \bl{}{
    L'interprete comandi è rappresentato dal programma \texttt{cmd.exe} in \texttt{C:$\backslash$Windows$\backslash$System32$\backslash$cmd.exe}
    \iz{
      \item Da non confondere con il vecchio inteprete (MS)DOS eseguito su macchina virtuale IA16...
      \item Eredita in realtà sintassi e funzionalità della maggior parte dei comandi del vecchio MSDOS
    }
  }
  \fg{height=0.4\textheight}{img/prompt.png}
}

\begin{frame}[fragile]{Aprire un terminale in laboratorio}
  \begin{itemize}
    \item In laboratorio, troverete il terminale (prompt dei comandi) clickando su Start $\Rightarrow$ Programmi $\Rightarrow$ Accessori $\Rightarrow$ Prompt dei comandi
    \item Metodo più rapido: Start  $\Rightarrow$ Nella barra di ricerca, digitare \texttt{cmd} $\Rightarrow$ clickare su \texttt{cmd.exe}
  \end{itemize}
\end{frame}

\begin{frame}[fragile]{File system e terminale: cheat sheet}
\label{slide:commands}
  \begin{center}
    \begin{tabular}{| l | c | c |}
      \hline
      \textbf{Operazione} & \textbf{Comando Unix} & \textbf{Comando Win} \\ \hline
      \scriptsize{}Visualizzare la directory corrente & \texttt{pwd} & \texttt{echo \%cd\%}  \\ \hline
      \scriptsize{}Eliminare il file \texttt{f} (non va con le cartelle!) & \texttt{rm f} & \texttt{del f} \\ \hline
      \scriptsize{}Eliminare la directory \texttt{nd} & \texttt{rm -r nd} & \texttt{rd nd} \\ \hline
      \scriptsize{}Contenuto della directory corrente & \texttt{ls -alh} & \texttt{dir} \\ \hline
%       \scriptsize{}Avviare un eseguibile di nome \texttt{pippo} & \texttt{./p} & \texttt{.\textbackslash{}p} \\ \hline
      \scriptsize{}Cambiare unità disco (passare a D:) & -- & \texttt{D:} \\ \hline
      \scriptsize{}Passare alla directory \texttt{nd} & \texttt{cd nd} & \texttt{cd nd} \\ \hline
      \scriptsize{}Passare alla directory di livello superiore & \texttt{cd ..} & \texttt{cd..} \\ \hline
      \scriptsize{}Spostare (rinominare) un file \texttt{f1} in \texttt{f2} & \texttt{mv f1 f2} & \texttt{move f1 f2} \\ \hline
      \scriptsize{}Copiare il file \texttt{f} in \texttt{fc} & \texttt{cp f fc} & \texttt{copy f fc} \\ \hline
      \scriptsize{}Creare la directory \texttt{d} & \texttt{mkdir d} & \texttt{md d} \\ \hline
    \end{tabular}
  \end{center}
  Eseguire delle prove ed esser certi di aver compreso come utilizzare ogni comando. Per \emph{cominciare} l'esame, in particolare, dovrete usare il comando \texttt{cd}: siate certi di aver capito cosa fa!
\end{frame}

\begin{frame}[fragile]{Uso intelligente del terminale}
  \begin{block}{Autocompletamento}
    \scriptsize{}
    Sia *nix che Windows offrono la possibilità di effettuare autocompletamento, ossia chiedere al sistema di provare a completare un comando. Per farlo si utilizza il tasto ``tab'' (quello con due frecce orientate in maniera opposta, sopra il lucchetto).
  \end{block}
  \begin{block}{Memoria dei comandi precendenti}
    \scriptsize{}
    Sia *nix che Windows offrono la possibilità di richiamare rapidamente i comandi inviati precedentemente premendo il tasto ``freccia su''. I sistemi *nix supportano anche il lancio di comandi eseguiti in sessioni precedenti (non perde memoria col riavvio del terminale). 
  \end{block}
  \begin{block}{Interruzione di un programma}
    \scriptsize{}
    È possibile interrompere forzatamente un programma (ad esempio perché inloopato). Per farlo, sia su Windows che in *nix, si prema ctrl+c.
  \end{block}
  \begin{block}{Ricerca nella storia dei comandi precedenti}
    \scriptsize{}
    Premendo ctrl+r seguito da un testo da cercare, i sistemi *nix supportano la ricerca all'interno dei comandi lanciati recentemente, anche in sessioni utente precedenti. Non disponibile su Windows.
  \end{block}
\end{frame}



\section{Lab Startup}

\fr{Preparazione Ambiente di Lavoro 1/3}{
	\iz{
		\item Accendere il PC
		\item Loggarsi sul sito del corso
		\iz{
			\item \textcolor{blue}{\url{https://elearning-cds.unibo.it/course/view.php?id=5844}}
		}
		\item Scaricare dalla sezione dedicata a questa settimana il file \texttt{lab01.zip} contenente il materiale dell'esercitazione odierna
		\item Spostare il file scaricato sul Desktop
		\item Decomprimere il file usando 7zip (o un programma analogo) sul Desktop
	}
}


\fr{Preparazione Ambiente di Lavoro 2/3}{
	\iz{
		\item Aprire il prompt dei comandi 
	}	
	\fg{height=0.8\textheight}{img/open_prompt.png}
}

\fr{Preparazione Ambiente di Lavoro 3/3}{
	\iz{
		\item Posizionarsi all'interno della cartella scompattata con l'ausilio del comando \texttt{cd} (change directory)
		\en{
			\item \texttt{cd Desktop} e premere invio, dopodiché
			\item \texttt{cd lab01} e premere invio
		}
	}
	\fg{height=0.4\textheight}{img/cd_prompt.png}
}

%====================
%Compilazione ed Esecuzione da Riga di Comando
%====================
\section{Compilazione ed Esecuzione da Riga di Comando}


\fr{Compilazione ed Esecuzione (base) 1/2}{
  \iz{
    \item Aprire il programma JEdit
    \iz{
      \item Si tratta di un editor di testo pensato per chi scrive codice
      \item Supporta vari linguaggi tramite plug-in
      \item Sviluppato in Java
      \item Multipiattaforma
      \item Sarà il nostro editor per questa prima lezione
      \item In futuro useremo un Integrated Development Environment (IDE)
    }
    \item Da JEdit aprire la classe \texttt{HelloWorld.java}
    \item Studiare brevemente il comportamento della classe prima di procedere con i passi successivi
  }
}


\fr{Compilazione ed Esecuzione (base) 2/2}{
	
	%\code{code/HelloWorld.java}
	
	\bl{}{
		Compilazione di una classe (comando \textcolor{blue}{javac})
		\iz{
			\item \alert{\cil{javac NomeSorgente.java}} 
			\item \alert{\cil{javac *.java}} compila tutti i sorgenti nella directory corrente
			\item Compiliamo \cil{HelloWorld.java} da riga di comando
				\iz {\item \cil{javac HelloWorld.java}}
		}
	}
	\bl{}{
		Esecuzione di un programma Java (comando \textcolor{blue}{java})
		\iz{
			\item	\alert{\cil{java NomeSorgente}}
			\item Eseguiamo \cil{HelloWorld} da riga di comando
				\iz {\item \cil{java HelloWorld}}
		}
	}
}


\fr{Warm-up: Prime Modifiche}{
	\bl{Modificare la classe HelloWorld seguendo i passi riportati qui sotto}{
		\en{
			\item Sostituire la stampa a video con un messaggio a piacere
			\item Ricompilare ed eseguire nuovamente il programma per verificarne il corretto comportamento
			\item Aggiungere alla fine del messaggio modificato il risultato della computazione 50*50
			\item Ricompilare ed eseguire nuovamente il programma per verificarne il corretto comportamento
		}
	}
	\bl{}{
		In caso di problemi/dubbi fare riferimento alle slide del modulo \texttt{02-Oggetti e classi}
	}
}

\section{Sviluppo e Test di Semplici Classi}

\subsection{Esercizi da Svolgere in Autonomia}

\fr{Modalità di Lavoro}{
	\bl{}{
		\en{
			\item Leggere la consegna
			\item Risolvere l'esercizio autonomamente, cercando di risolvere da soli eventuali piccoli problemi che possono verificarsi durante lo svolgimento degli esercizi
			\item Contattare i docenti nel caso vi troviate a lungo bloccati nella risoluzione di uno specifico esercizio
			\item A esercizio ultimato contattare i docenti per un rapido controllo della soluzione realizzata
			\item Proseguire con l'esercizio seguente
		}
	}
}


\fr{Test Scope delle Variabili}{
	%\sizedcode{\scriptsize}{code/TestScopes.java}
	\bl{}{
		\en{ 
			\item Analizzare il sorgente \cil{TestScopes.java}
			\item Prima di compilare e lanciare il programma riflettere sul comportamento dei metodi della classe e provare a prevedere l'output di ogni singola stampa
			\item Compilare ed eseguire il programma. L'output ottenuto corrisponde a quanto preventivato?
			\item In caso di risultato diverso da quanto aspettato (in questo caso e in tutti i successivi)
				\en{
					\item Provare a ragionare e a trovare in autonomia una spiegazione 
					\item Se non è possibile risolvere i dubbi in maniera autonoma, contattare il docente
				}
		}
	}
}

	
\fr{Costruzione di Semplici Classi 1/6}{
	\bl{}{
		Completare la classe \cil{Student} caratterizzata da
		\iz{
			\item Campi
			\iz {
				\item \cil{String name}
				\item \cil{String surname}
				\item \cil{int id} (rappresenta la matricola)
				\item \cil{int matriculationYear}
			}
			\item Metodi da implementare
			\iz {
				\item \cil{build} (inizializza \cil{Student} con i parametri forniti in input)
				\iz {
					\item Input: \cil{String name, String surname, int id, int matriculationYear}
				}
				\item \cil{printStudentInfo} (stampa in standard output le informazioni relative allo studente)
			}
		}
	}
	\bl{}{
		Infine, testare la classe completando il \cil{main} secondo le linee guida riportate nei commenti del sorgente
	}
}

\fr{Costruzione di Semplici Classi 2/6}{
	\bl{}{
		Completare la classe \cil{ComplexNum} caratterizzata da
		\iz{
			\item Campi
			\iz {
				\item \cil{double re} (parte reale), \cil{double im} (parte immaginaria)
			}
			\item Metodi (oltre al main)
			\iz {
				\item \cil{build} (inizializza \cil{ComplexNum} con i paramtri forniti in input)
				\iz {
					\item Input: \cil{double re, double im}
				}
				\item equal (determina se il numero complesso è = a quello fornito in input)
				\iz {
					\item Input: \cil{ComplexNum num}
					\item Returns: \cil{boolean} 
				}
				\item add (aggiunge il \cil{ComplexNum} fornito in input)
				\iz {
					\item Input: \cil{ComplexNum num}
				}
				\item \cil{toStringRep} (restituisce una rappresentazione testuale del numero)
				\iz {
					\item Returns: \cil{String}
				}
			}
		}
	}
	\bl{}{
		Completare e testare la classe seguendo le linee guida riportate nei commenti del sorgente
	}
}

\fr{Costruzione di Semplici Classi 3/6}{
	\bl{}{
		Implementare da zero la classe \texttt{Calculator} che realizzi il comportamento di una semplice calcolatrice in grado di lavorare su numeri \cil{double}
	}
	\bl{Caratteristiche della classe \cil{Calculator}:}{
		\iz{
			\item Campi: - (nessuno) (il ch\'e la rende una classe ``degenere'' dal punto di vista dell'OOP)
			\item Metodi (oltre al \texttt{main})
			\iz{
				\item \cil{sum/sub/mul/div}
					\iz{
						\item Input: \cil{double n1, double n2}
						\item Returns: \cil{double} (risultato dell'operazione)
					}
			}
		}
	}
	\bl{Testing della classe}{
		Infine, testare la classe completando il \cil{main} secondo le linee guida riportate nella slide che segue
	}
}

\fr{Costruzione di Semplici Classi 3/6}{
	\bl{Test della classe \texttt{Calculator}}{
		\sizedcode{\tiny}{code/testCalculator.txt}
	}
}


\fr{Costruzione di Semplici Classi 4/6}{
	\bl{Modificare la classe \texttt{Calculator} costruita in precedenza}{
		\iz{
			\item Aggiungere due campi
				\iz{
					\item \cil{int nOpDone} (n. di operazioni compiute dalla calcolatrice)
					\item \cil{double lastRes} (ultimo risultato computato)
				}
			\item Inizializzare i campi a zero nel metodo \cil{build}
				\iz{
					\item Il metodo dovrà avere zero parametri di input
				}
			\item Modificare i metodi \cil{add/sub/mul/div} tenendo conto dei due nuovi campi aggiunti alla classe 
		}
	}
	\bl{}{
		Verificare il corretto funzionamento delle modifiche utilizzando la calcolatrice come in precedenza e stampando di tanto in tanto in standard output il valore dei campi \cil{nOpDone} e \cil{lastRes}
	}
}

\fr{Costruzione di Semplici Classi 5/6}{
	\bl{}{
		Implementare le classi \cil{UseComplex} e \cil{UseCalculator} composte dal solo metodo \cil{main} in cui testare il corretto comportamento di \cil{ComplexNum} e \cil{Calculator} realizzate in precedenza
	}
			
	\bl{Esempi di possibili test}{
		\iz{
			\item Costruire e calcolare la somma di più numeri complessi
			\item Verificare il corretto comportamento del metodo \cil{equal} della classe \cil{ComplexNum}
			\item Verificare il corretto funzionamento dei vari metodi forniti dalla classe \cil{Calculator} per diversi valori di input
			\item Provare ad eseguire il metodo \cil{div} fornendo come valore per il secondo parametro 0. Cosa succede?
			\item ...
		}
	}
}

\fr{Costruzione di Semplici Classi 6/6 (segue)}{
	\bl{}{
		Implementare una classe Java che modelli il concetto di treno
	}
	\bl{Caratteristiche della classe \cil{Train}:}{ 
		\iz{
			\item Campi
			\iz{
				\item \cil{int nTotSeats} (n. posti totale = nFCSeats+nSCSeats)
				\item \cil{int nFCSeats/nSCSeats} n. posti in prima/seconda (FC/SC) classe
				\item \cil{int nFCReservedSeats/nSCReservedSeats} n. di posti prenotati in prima/seconda classe
			}
			\item Metodi (prevedere \cil{build} per l'inizializzazione dei campi)
			\iz{
				\item \cil{reserveFCSeats/reserveSCSeats}
				\iz{
					\item Input: \cil{int nSeats}
				}
				\item \cil{getTotOccupancyRatio}
				\iz{
					\item Returns: \cil{double} (percentuale globale di posti occupati)
				}
				\item \cil{getFCOccupancyRatio/getSCOccupancyRatio}
				\iz{
					\item Returns: \cil{double} (percentuale posti occupati in FC/SC)
				}
				\item \cil{deleteAllReservations} (azzera tutti i posti prenotati)
			}
		}
	}
}

\fr{Costruzioni di Semplici Classi 6/6 (seguito)}{
	\bl{Testing}{
		\iz{
			\item Da realizzare all'interno della classe \cil{UseTrain}, fornita tra il materiale di questa prima esercitazione 
			\item Seguire le linee guida fornite nei commenti del \cil{main}
			\item Utilizzare sempre valori numerici consistenti con i vincoli scelti (vedere commenti)
		}
	}
	\bl{Nota}{
		Gestire la conversione da \cil{int a double} come siete abituati a fare in C per quanto riguarda il calcolo delle percentuali di posti occupati
	}		
}
\subsection{Approfondimento: Passaggio dei Parametri}


\fr{Richiamo dalla Teoria}{
	\bl{Passaggio dei Parametri}{
		Java gestisce in maniera diversa dal C il passaggio dei parametri
		\iz{
			\item In particolare
			\iz{
				\item Tutti i tipi di dato primitivi (\cil{int, long, double}, etc.) sono passati per valore
				\item Gli oggetti invece sono \textcolor{blue}{\textit{sempre}} passati per riferimento
			}
			\item E' importante tenere conto di questo aspetto durante la realizzazione di programmi Java per non incorrere in side-effect
		}
	}
}

\fr{Test Passaggio Per Valore}{
	\sizedcode{\scriptsize}{code/CallByValueExample.java}
	\bl{}{
		\en{
			\item Analizzare la classe \cil{CallByValueExample}
			\item Cercare di prevedere l'output
			\item Compilare ed eseguire il programma. L'output ottenuto corrisponde a quanto preventivato?
			\item In caso di risultato diverso da quanto aspettato contattare il docente
		}
	}
}



\fr{Point3D}{
	\bl{}{
		Nel prossimo esempio si utilizzerà la classe \cil{Point3D} già vista a lezione
		\iz{
			\item Quindi, prima di procedere, compilarla con il comando
			\iz{
				\item \texttt{javac Point3D.java}
			}
		}
	}
}

\fr{Test Passaggio Per Riferimento 1/2}{
	\sizedcode{\tiny}{code/CallByReferenceExample1.java}
	\bl{}{
		\en{
			\item Analizzare la classe \cil{CallByReferenceExample1}
			\item Cercare di prevedere l'output
			\item Compilare ed eseguire il programma. L'output ottenuto corrisponde a quanto preventivato?
			\item In caso di risultato diverso da quanto aspettato contattare il docente
		}
	}
}

\fr{Test Passaggio Per Riferimento 2/2}{
	\sizedcode{\tiny}{code/CallByReferenceExample2.java}
	\bl{}{
		\en{
			\item Analizzare la classe \cil{CallByReferenceExample2}, cercando di prevederne l'output
			\item Compilare ed eseguire il programma 
			\item L'output ottenuto presenta ``side-effect''? Qual è la spiegazione per il risultato ottenuto?
			\item Per chiarimenti e approfondimenti contattare il docente
		}
	}
}

\section{Esercizi Aggiuntivi}

\fr{Una Calcolatrice per Numeri Complessi}{
	
	\bl{}{
		Completare e testare una classe Java che realizzi il comportamento di una semplice calcolatrice in grado di lavorare su numeri complessi rappresentati dalla classe \cil{ComplexNum}
	}
	\bl{Caratteristiche della classe \cil{ComplexNumCalculator}:}{
		\iz{
			\item Campi \cil{int nOpDone, ComplexNum lastRes}
			\item Metodi (oltre ai soliti \cil{main} e \cil{build})
			\iz{
				\item \cil{sum/sub/mul}
				\iz{
					\item Input: \cil{ComplexNum n1, ComplexNum n2}
					\item Returns: \cil{ComplexNum} (risultato dell'operazione)
				}
			}
		}
	}
}

\fr{Un Riconoscitore della Sequenza Ripetuta 1234}{
	\bl{Classe \texttt{Recognizer}}{
		Completare e testare una classe Java che sia in grado di riconoscere la sequenza 1234 ripetuta (potenzialmente all'infinito)
	}
	\bl{Traccia Risolutiva}{
		\iz{
			\item Implementare un metodo diverso per gestire il riconoscimento di ciascun carattere (\texttt{check1}, \texttt{check2}, \texttt{check3},..)
			\item Tenere traccia all'interno della classe delle informazioni di stato relative al prossimo carattere da riconoscere
            \item Implementare il metodo \cil{nextExpectedInt} che restituisce il prossimo intero atteso nella sequenza
			\item Implementare il metodo \cil{reset} che consente di resettare lo stato corrente del riconoscitore 
		}
	}
}
	

\begin{frame}[allowframebreaks]
 \frametitle{Bibliography}
	\bibliographystyle{plain}
	\small
 \bibliography{biblio}
\end{frame}


\end{document}


%====================
%Java come Linguaggio Object-Oriented
%====================
\fr{Java come Linguaggio Object-Oriented}{
	\bl{Un po' di storia}{Java nasce nei laboratori della Sun microsystem per l'embedded networking computing agli inizi degli anni '90
		\iz{
			\item Handheld home-entertainment controller: primo \textit{testbed} per il linguaggio, pensato per la TV via cavo
				\iz{\item Abbandonato perché troppo complesso per quello specifico contesto}
			\item Poi applicato - inizialmente - in ambito web (Netscape browser, 1995)
			\item Oggi uno dei linguaggi più usati al mondo
			\item Per maggiori dettagli sulla storia del linguaggio \cite{java-history-wiki,java-early}
		}
	}
	
	\bl{(Almost) Everything is an Object}{
		\iz{
			\item Tranne i tipi primitivi base (\cil{int}, \cil{float}, ...)...
			\item .. esistono solo \alert{classi} (a livello statico) e \alert{oggetti} (a runtime)
%				\iz{
%					\item Anche il \textit{main} è definito dentro a una classe!
%				}
		}
	}
}

