%\documentclass[12pt,handout]{beamer}
\documentclass[presentation]{beamer}
\usepackage{../oop-slides-lab}
\setbeamertemplate{bibliography item}[text]

\newcommand{\lab}{Lab02}

\title[{\lab} -- Strumenti Avanzati]{Stile del Codice, Compilazione/Esecuzione avanzata di programmi Java, Programmi con Argomenti}

\date[\today]{\today}

\begin{document}

\frame[label=coverpage]{\titlepage}

%====================
%Outline
%====================
\begin{frame}<beamer>
	\frametitle{Outline}
	\tableofcontents[]
\end{frame}

\fr{Preparazione ambiente di lavoro}{
\begin{itemize}\itemsep10pt
\item Accedere al PC di laboratorio con le proprie credenziali istituzionali
		\item Accedere al sito del corso
%	\iz{
%		\item \textcolor{blue}{\url{http://bit.ly/oop19-cesena}}
%	}
	\item Scaricare il materiale dell'esercitazione odierna
	\item Spostare il file scaricato sul Desktop
	\item Decomprimere il file e aprire con Visual Studio Code la directory ottenuta (File $\to$ Open Folder...)
	\item Puntare il terminale alla directory con i sorgenti dell'esercitazione odierna
\end{itemize}
}

\section{Stili e Convenzioni per il codice sorgente}

\begin{frame}[allowframebreaks]{Stili e Convenzioni}
	\begin{itemize}\itemsep10pt
		\item Il codice sorgente che un programmatore scrive, generalmente è condiviso con altre persone (del proprio team, ma anche persone esterne al team o la community)
		\begin{itemize}
			\item è importante scrivere software immediatamente comprensibile
			\item il fatto che un software ``giri'' (rispetti i requisiti e/o produca i risultati attesi) non è una sufficente metrica di qualità
		\end{itemize}
		\item è importante, fondamentale, adottare uno stile e seguirlo
		\begin{itemize}
			\item chiaro
			\item condiviso
			\item consistente
		\end{itemize}
	\end{itemize}

\begin{block}{}
\begin{quote}
Always code as if the guy who ends up maintaining your code will be a violent psychopath who knows where you live. Code for readability.
\flushright{--- John Woods [disputed]}
\end{quote}
\end{block}


\begin{block}{Ogni linguaggio ha le sue prassi...}
\begin{itemize}
		\item Quelle Java di riferimento sono disponibili qui:
		\tiny
		\begin{itemize}
			\item \url{http://bit.ly/java-style-guide}
			\item \url{http://bit.ly/java-code-conventions}
			\item \url{http://bit.ly/oracle-java-code-conventions}
		\end{itemize}
	\end{itemize}
\end{block}

\begin{block}{Ogni azienda poi è libera di darsi altre regole interne}

\begin{itemize}
\item Ad esempio:
\footnotesize
	\begin{itemize}
		\item Google: \url{http://archive.is/a0Jhz}
 		\item Twitter: \url{http://archive.is/aa1tE}
 		\item Mozilla: \url{http://archive.is/rs3Ns}
	\end{itemize}
	\normalsize
\item Notare che sono sempre \textbf{consistenti}!
\begin{itemize}
\item E che sono tipicamente restrizioni delle convenzioni, non modifiche!
\end{itemize}
\end{itemize}
\end{block}

\begin{center}
\textbf{Nel corso faremo riferimento alle Java Code Conventions\\(con qualche vincolo in più)}
\end{center}
\end{frame}

\begin{frame}[allowframebreaks]{Java Code Conventions (un estratto)}
	\begin{block}{Usare sempre le parentesi per if, else, for, while, anche se segue una sola istruzione}
		\begin{itemize}
			\item Aumentano la manutenibilità del codice
			\item È facile che nella fretta si modifichi il codice in modo sbagliato
			\item È facile che alcuni tool automatici si sbaglino quando ``uniscono'' pezzi di codice scritti da diverse persone
			\item Apple iOS soffrì di un grave bug a SSL/TLS causato da questa cattiva pratica \url{http://archive.is/KQp8E}
		\end{itemize}
	\end{block}
	\begin{block}{Le parentesi graffe vanno sempre ``all'egiziana'' (Egyptian brackets)}
		\begin{itemize}
			\item La graffa che apre va in linea con lo statement di apertura
			\item La graffa che chiude va in a capo, nella stessa colonna dello statement di apertura
		\end{itemize}
	\end{block}
	\begin{block}{Naming conventions - \textbf{molto importanti!}}
		\begin{itemize}
			\item I nomi di \texttt{package} usano sempre e solo lettere minuscole
			\item Usare sempre camelCase, evitare gli underscore (\_)
			\item I nomi di classe cominciano sempre per maiuscola
			\item I nomi di campi, metodi e variabili locali iniziano sempre per minuscola
			\item I campi \texttt{static final} (costanti di classe) sono interamente maiuscoli e possono usare underscore
		\end{itemize}
	\end{block}
	
Come seguire stili e convenzioni? Tutto nelle mani del programmatore?
\begin{itemize}
\item Esistono strumenti automatici a supporto, introdotti nelle prossime lezioni...
	\end{itemize}
\end{frame}


\section{Compilazione ed esecuzione avanzata in Java}

\fr{Nuova opzione per \texttt{javac}}{
	\begin{itemize}
		\item Abbiamo già visto come compilare file sorgenti Java (file \texttt{.java}), generando classi in bytecode, che prendono la forma di file \texttt{.class} nella medesima directory 
		\item Tuttavia è uso comune e \textbf{buona pratica} nella gestione di progetti articolati, \textbf{separare le classi sorgenti dal bytecode}, ad esempio:
	\begin{itemize}
		\item cartella \texttt{src}, per i file sorgenti (\texttt{.java})
		\item cartella \texttt{bin}, contenente le classi compilate (\texttt{.class})
	\end{itemize}
		\item Come si fa?
	\end{itemize}
	\bl{Nuova opzione del comando \texttt{javac}}{
		\iz {
			\item \cil{-d}: consente di specificare la cartella destinazione in cui compilare i file \texttt{.java}
			\item Si tratta di un'opzione che dovete obbligatoriamente saper usare
		}
		\center{\textbf{Sarà oggetto di valutazione in sede di prova pratica!}}
	}
}

\begin{frame}{Compilazione di più file da qualunque directory verso una qualunque directory}
  \begin{block}{Compilazione in directory arbitrarie}
    \texttt{javac -d <CARTELLA DESTINAZIONE> <FILE JAVA>}
    \begin{itemize}
      \item \textbf{OVVIAMENTE} vanno sostituite le variabili fra parentesi angolari con le 
directory che andranno usate.
    \end{itemize}
  \end{block}
  \begin{block}{Compilazione di più file in una singola passata}
    \texttt{javac -d <CARTELLA DESTINAZIONE> <ELENCO DI FILE JAVA>}
    \begin{itemize}
      \item \textbf{OVVIAMENTE} vanno sostituite le variabili fra parentesi angolari con le 
directory che andranno usate.
    \end{itemize}
  \end{block}
  È possibile anche utilizzare la wildcard (\texttt{*}) invece di elencare tutti i file!
  \begin{itemize}
    \item Su Unix si possono usare wildcard in più punti del path, ad esempio 
\texttt{progetti/*/src/*.java} elenca tutti i file con estensione java dentro ciascuna cartella 
\texttt{src} di ciascuna cartella dentro \texttt{progetti}
  \end{itemize}
\end{frame}

\begin{frame}[allowframebreaks]{Il classpath in Java}
	\begin{itemize}
		\item Il risultato della compilazione di sorgenti Java sono una o più \textbf{classi}
		\item Quando si va ad eseguire, si eseguono \textbf{classi}, non files
		\begin{itemize}
			\item Infatti la virtual machine si aspetta il nome completo di una classe in input
			\begin{itemize}
				\item Non il file dov'è scritta
				\item Né il file dov'è compilata
			\end{itemize}
		\end{itemize}
	\end{itemize}
	\begin{block}{Come fa la JVM a risolvere le classi?}
		\begin{itemize}
			\item Possiede un elenco di percorsi che contengono i file compilati
			\begin{itemize}
				\item All'interno di questi percorsi, i file devono essere opportunamente organizzati
				\item Ad esempio, la struttura delle cartelle deve replicare quella dei package
			\end{itemize}
			\item Cerca (in ordine) nei suddetti percorsi la classe che gli serve
			\item I percorsi possono essere directory, file compressi, o indirizzi di rete
			\item Per approfondire: \url{http://archive.is/0ziau}
		\end{itemize}
	\end{block}
	\begin{center}
		L'insieme ordinato dei percorsi prende il nome di \textbf{classpath}
	\end{center}
	\begin{block}{Default classpath}
		Se non specificato, il classpath di Java include automaticamente:
		\begin{itemize}
			\item I file jar del Java Runtime Environment
			\begin{itemize}
				\item \texttt{rt.jar} ed altri file importanti
				\item Contengono ad esempio \texttt{java.lang.Math}
			\end{itemize}
			\item La directory corrente
		\end{itemize}
	\end{block}
	\begin{block}{Aggiungere directory al classpath}
		Possono essere aggiunte directory al classpath
		\begin{itemize}
			\item Si usa l'opzione \texttt{-cp} (o, equivalentemente, \texttt{-classpath}), seguita da un elenco di percorsi
			\begin{itemize}
				\item separati dal simbolo \texttt{:} (Unix)
				\item o dal simbolo \texttt{;} (Windows)
			\end{itemize}
		\end{itemize}
	\end{block}
\end{frame}

\begin{frame}{Più cartelle nel classpath}
	Il classpath non è composto da una sola cartella, ma può contenere più cartelle, oltre a file compressi contenenti bytecode e risorse, come zip o jar (li vedremo in futuro).
	\begin{block}{}
		Si possono specificare più cartelle utilizzando come separatore il simbolo ``\texttt{:}'' (per sistemi Unix) oppure ``\texttt{;}'' (per sistemi Windows)
		\begin{itemize}
			\item \texttt{javac -d bin -cp lib1:lib2:lib3 src/*.java}
			\begin{itemize}
				\item Compila tutti i file con estensione java che si trovano nella cartella \texttt{src}, mettendo i compilati dentro \texttt{bin}. In compilazione, potrà linkare tutte le classi che si trovano nelle cartelle \texttt{lib1}, \texttt{lib2} e \texttt{lib3}: nel caso in cui alcuni sorgenti in src stiano usando delle classi definite dentro queste cartelle, la compilazione avrà successo.
				\item \scriptsize{Equivalente Windows: \texttt{javac -d bin -cp lib1;lib2;lib3 src/*.java}}
		\end{itemize}
			\item \texttt{java -cp bin:lib1:lib2:lib3 MyClass}
			\begin{itemize}
				\item Esegue il main della classe \texttt{MyClass}. Cercherà questa classe e tutte quelle collegate all'interno delle cartelle \texttt{bin}, \texttt{lib1}, \texttt{lib2} e \texttt{lib3}.
				\item \scriptsize{Equivalente Windows: \texttt{java -cp bin;lib1;lib2;lib3 MyClass}}
			\end{itemize}
		\end{itemize}
	\end{block}
\end{frame}

\begin{frame}{Organizzazione dei sorgenti in presenza di package}
        È buona norma organizzare i sorgenti in modo da rappresentare su filesystem la struttura 
dei package. Si noti però che (dato che il compilatore lavora su \alert{file}) questa scelta 
\textbf{non è \underline{teoricamente} obbligatoria}!
    \begin{itemize}
        \item Lo è di fatto in questo corso, perché le cose van fatte bene
        \item Lo sarà nel mondo del lavoro, perché è prassi assolutamente comune
    \end{itemize}
    \begin{block}{Risultato della compilazione}
        Quando viene compilato un sorgente la cui classe è contenuta in un package, il compilatore 
riproduce la struttura dei package usando delle directory
        \begin{itemize}
            \item Dato che l'interprete non lavora con file ma con \alert{classi}, il loro layout 
sul file system \textbf{non può essere modificato!} 
        \end{itemize}
    \end{block}
    \begin{block}{Esecuzione}
        L'esecuzione è identica al caso precedente, si faccia solo attenzione ad 
usare l'intero nome della classe, che in Java include anche il nome del package!
    \end{block}
\end{frame}

\begin{frame}[fragile, allowframebreaks]{Uso del classpath in fase di compilazione}
   Supponiamo di avere in mano la seguente classe:
    \sizedcode{\tiny}{../../workspace/lab02/02-classpathcompilation/src/oop/lab02/math/UseComplex.java}
    \begin{block}{Comprensione degli errori}
        Se provassimo a compilarla, otterremmo degli errori
        \sizedcode{\tiny}{code/esempio3/src/oop/lab02/math/error.txt}
        \begin{itemize}
		\item Il compilatore ha bisogno di conoscere la classe \texttt{ComplexNum} per poterla 
linkare e per poter compilare una classe che la riferisce
		\item Il compilatore cerca nel classpath il bytecode della classe \texttt{ComplexNum}
		\item Come risolviamo?
        \end{itemize}
    \end{block}
    \begin{block}{Utilizzo di \texttt{-cp} in fase di compilazione}
        \begin{itemize}
            \item Supponiamo di avere solo la versione compilata di \texttt{ComplexNum} (ovvero non il sorgente)
            \begin{itemize}
                \item Notate che questa è la \textit{norma} quando si usano delle librerie: vengono 
                fornite già compilate!
            \end{itemize}
            \item Basterà mettere il percorso in cui \texttt{ComplexNum} è stata compilata nel 
            classpath di \texttt{javac}!
            \item Supponiamo di avere \texttt{UseComplex.java} nel percorso 
            \texttt{src/oop/lab02/math/}
            \item Supponiamo di aver \texttt{ComplexNum} compilato dentro \texttt{lib/}
            \item Possiamo usare:
        \end{itemize}
        \alert{\texttt{javac -d bin -cp lib src/oop/lab02/math/UseComplex.java}}
    \end{block}
    \begin{block}{Spiegazione del comando}
        \texttt{javac -d bin -cp lib src/oop/lab02/math/UseComplex.java}
        \begin{itemize}
            \item \texttt{javac} $\Rightarrow$ Invocazione del compilatore
            \item \texttt{-d bin} $\Rightarrow$ \texttt{-d} determina la \textbf{destinazione}. 
Vogliamo compilare dentro la cartella \texttt{bin}
            \item \texttt{-cp lib} $\Rightarrow$ \texttt{-cp} consente di aggiungere percorsi al 
            \textbf{classpath}. Noi vogliamo cercare le classi che ci servono, oltre che nella 
            posizione corrente e nelle librerie java, anche dentro \texttt{lib}
            \item \texttt{src/oop/lab02/math/UseComplex.java} $\Rightarrow$ Il \textit{file} che 
vogliamo compilare
        \end{itemize}
    \end{block}
\end{frame}

\begin{frame}{Passare più percorsi al classpath}
    Avendo come riferimento l'esempio precedente, proviamo ad eseguire.
    \begin{itemize}
    \item Per eseguire correttamente \texttt{UseComplex} dobbiamo dire alla JVM, tramite 
\textbf{-cp}, dove trovare:
    \begin{itemize}
        \item \texttt{ComplexNum}
        \item \texttt{UseComplex}
    \end{itemize}
        \item Si trovano in \textit{due percorsi diversi}!
        \item Dobbiamo specificare come argomento di \texttt{-cp} due percorsi, usando il 
\textbf{separatore}:
        \begin{itemize}
            \item \textbf{\texttt{:}} su sistemi UNIX (Linux, Mac OSX, BSD)
            \item \textbf{\texttt{;}} su sistemi Windows
        \end{itemize}
        \item Useremo quindi:
        \begin{itemize}
            \item \texttt{java -cp bin:lib oop.lab02.math.UseComplex} (Unix)
            \item \texttt{java -cp bin;lib oop.lab02.math.UseComplex} (Windows)
        \end{itemize}
    \end{itemize}
\end{frame}

\begin{frame}{Consiglio finale}
  Visto che all'esame il loro utilizzo è richiesto, è necessario imparare \textbf{a memoria} le opzioni di \texttt{java} e \texttt{javac}?
  \begin{center}
    \textbf{NO}
  \end{center}
  Entrambi i comandi (e praticamente tutti i comandi Unix) hanno con loro un'opzione che consente di stampare a video un help. Provate
  \begin{block}{}
    \begin{itemize}
      \item java -help
      \item javac -help
    \end{itemize}
  \end{block}
  \vspace{10pt}
  
  Gli help stampano abbondante testo con le relative istruzioni e a me serve una riga, davvero devo imparare a leggere e capire un help?
  \begin{center}
    \textbf{SÌ}
  \end{center}
  È molto facile dimenticarsi la sintassi delle opzioni di comandi che non si usano spesso. È molto più facile imparare a destreggiarsi in un help che andare a tentativi o ricordare cose a memoria.
\end{frame}

\section{Esecuzioni di programmi java con argomenti}

\begin{frame}[allowframebreaks]{Passaggio di argomenti ad un programma Java}
	\begin{itemize}\itemsep10pt
		\item La maggior parte dei comandi supporta degli argomenti
		\begin{itemize}
			\item Ad esempio, quando eseguite \texttt{javac -d bin MyClass.java} gli argomenti sono:
			\begin{enumerate}
			 \item \texttt{-d}
			 \item \texttt{bin}
			 \item \texttt{MyClass.java}
			\end{enumerate}
		\end{itemize}
		\item In C, questi vengono passati al metodo \texttt{main()} come coppia di \texttt{char **} e \texttt{int}, rappresentanti rispettivamente un riferimento all'area di memoria dove sono salvati i parametri ed il numero dei suddetti.
		\item Anche in Java ovviamente è possibile passare degli argomenti ad un programma 
		\end{itemize}

\framebreak		
		
	\begin{itemize}\itemsep10pt
	\item La gestione è un po' più semplice che in C, grazie al fatto che gli array si portano dietro l'informazione circa la loro dimensione
		\item E grazie al fatto che la signature del metodo \texttt{main()} è una sola in Java
		\begin{itemize}
			\item \texttt{public static void main(String [])} è l'unica signature valida
			\item Mentre in C sia \texttt{int main(void)} che \texttt{int main(char **, int)} sono ugualmente accettabili
		\end{itemize}
	
	
		\item Gli argomenti con cui un programma Java viene invocato vengono passati come parametri attraverso l'array (\texttt{String[] args}) che il metodo \texttt{main()} prende in ingresso
		\begin{itemize}
		\item Nonostante sia un parametro del \emph{metodo principale} di qualunque programma Java, si tratta di un comune array senza alcuna particolarità.
		\end{itemize}
		
	\end{itemize}
\end{frame}

\section{Appendice: richiami utili per gli esercizi del {\lab}}

\begin{frame}{A1 -- Varianza}

\begin{block}{Formula per il calcolo della varianza}
Sia $n$ il numero di elementi dell'array ed $x_i$ l'elemento all'indice $i$ dell'array, e $\mu$ la media dei valori del suddetto array. La varianza $\sigma^2$ può essere calcolata come:

\centering
\huge
$\sigma^2 = \frac{\displaystyle\sum_{i=0}^{n-1}(x_i - \mu)^2} {n}$
\end{block}
\end{frame}

%\begin{frame}[allowframebreaks]
% \frametitle{Bibliography}
%	\bibliographystyle{plain}
%	\small
% \bibliography{biblio}
%\end{frame}


\end{document}
